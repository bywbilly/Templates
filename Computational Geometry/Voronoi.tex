\begin{lstlisting}

#define Oi(e) ((e)->oi)
#define Dt(e) ((e)->dt)
#define On(e) ((e)->on)
#define Op(e) ((e)->op)
#define Dn(e) ((e)->dn)
#define Dp(e) ((e)->dp)
#define Other(e, p) ((e)->oi == p ? (e)->dt : (e)->oi)
#define Next(e, p) ((e)->oi == p ? (e)->on : (e)->dn)
#define Prev(e, p) ((e)->oi == p ? (e)->op : (e)->dp)
#define V(p1, p2, u, v) (u = p2->x - p1->x, v = p2->y - p1->y)
#define C2(u1, v1, u2, v2) (u1 * v2 - v1 * u2)
#define C3(p1, p2, p3) ((p2->x - p1->x) * (p3->y - p1->y) - (p2->y - p1->y) * (p3->x - p1->x))
#define Dot(u1, v1, u2, v2) (u1 * u2 + v1 * v2)
#define dis(a,b) (sqrt( (a->x - b->x) * (a->x - b->x) + (a->y - b->y) * (a->y - b->y) ))

const int maxn = 110024;
const double eps = 1e-7;
const int aix = 4;
int n, M, k;

struct gEdge {
  int u, v;
  double w;
  bool operator <(const gEdge &e1) const {
    return w < e1.w - eps;
  }
} E[aix * maxn], MST[maxn];

int b[maxn];
int Find(int x) {
  while (x != b[x]) {
    b[x] = b[b[x]];
    x = b[x];
  }
  return x;
}

void Kruskal() {
  int m1, m2;
  memset(b, 0, sizeof(b));
  for (int i = 0; i < n; i++) b[i] = i;
  sort(E, E + M);
  for (int i = 0, kk = 0; i < M && kk < n - 1; i++) {
    m1 = Find(E[i].u);
    m2 = Find(E[i].v);
    if (m1 != m2) {
      b[m1] = m2;
      MST[kk++] = E[i];
    }
  }
}

struct point {
  double x, y;
  int index;
  struct edge *in;
  bool operator <(const point &p1) const {
    return x < p1.x - eps || (abs(x - p1.x) <= eps && y < p1.y - eps);
  }
};

struct edge {
  point *oi, *dt;
  edge *on, *op, *dn, *dp;
};

point p[maxn], *Q[maxn];
edge mem[aix * maxn], *elist[aix * maxn];
int nfree;

//memory
void Alloc_memory() {
  nfree = aix * n;
  edge *e = mem;
  for (int i = 0; i < nfree; i++)
    elist[i] = e++;
}

//Add an edge to a ring of edges
void Splice(edge *a, edge *b, point *v) {
  edge *next;
  if (Oi(a) == v)
    next = On(a), On(a) = b;
  else
    next = Dn(a), Dn(a) = b;
  if (Oi(next) == v)
    Op(next) = b;
  else
    Dp(next) = b;
  if (Oi(b) == v)
    On(b) = next, Op(b) = a;
  else
    Dn(b) = next, Dp(b) = a;
}

//Initialise a new edge
edge *Make_edge(point *u, point *v) {
  edge *e = elist[--nfree];
  e->on = e->op = e->dn = e->dp = e;
  e->oi = u;
  e->dt = v;
  if (!u->in) u->in = e;
  if (!v->in) v->in = e;
  return e;
}

//Creates a new edge and adds it to two rings of edges.
edge *Join(edge *a, point *u, edge *b, point *v, int side) {
  edge *e = Make_edge(u, v);
  if (side == 1) {
    if (Oi(a) == u)
      Splice(Op(a), e, u);
    else
      Splice(Dp(a), e, u);
    Splice(b, e, v);
  } else {
    Splice(a, e, u);
    if (Oi(b) == v)
      Splice(Op(b), e, v);
    else
      Splice(Dp(b), e, v);
  }
  return e;
}

//Remove an edge
void Remove(edge *e) {
  point *u = Oi(e), *v = Dt(e);
  if (u->in == e)
    u->in = e->on;
  if (v->in == e)
    v->in = e->dn;
  if (Oi(e->on) == u)
    e->on->op = e->op;
  else
    e->on->dp = e->op;
  if (Oi(e->op) == u)
    e->op->on = e->on;
  else
    e->op->dn = e->on;
  if (Oi(e->dn) == v)
    e->dn->op = e->dp;
  else
    e->dn->dp = e->dp;
  if (Oi(e->dp) == v)
    e->dp->on = e->dn;
  else
    e->dp->dn = e->dn;
  elist[nfree++] = e;
}

//Determines the lower tangent of two triangulations
void Low_tangent(edge *e_l, point *o_l, edge *e_r, point *o_r, edge **l_low, point **OL, edge **r_low, point **OR) {
  point *d_l = Other(e_l, o_l), *d_r = Other(e_r, o_r);
  while (1) {
    if (C3(o_l, o_r, d_l) < -eps) {
      e_l = Prev(e_l, d_l);
      o_l = d_l;
      d_l = Other(e_l, o_l);
    } else if (C3(o_l, o_r, d_r) < -eps) {
      e_r = Next(e_r, d_r);
      o_r = d_r;
      d_r = Other(e_r, o_r);
    } else
      break;
  }
  *OL = o_l, *OR = o_r;
  *l_low = e_l, *r_low = e_r;
}

void Merge(edge *lr, point *s, edge *rl, point *u, edge **tangent) {
  double l1, l2, l3, l4, r1, r2, r3, r4, cot_L, cot_R, u1, v1, u2, v2, n1, cot_n, P1, cot_P;
  point *O, *D, *OR, *OL;
  edge *B, *L, *R;
  Low_tangent(lr, s, rl, u, &L, &OL, &R, &OR);
  *tangent = B = Join(L, OL, R, OR, 0);
  O = OL, D = OR;
  do {
    edge *El = Next(B, O), *Er = Prev(B, D), *next, *prev;
    point *l = Other(El, O), *r = Other(Er, D);
    V(l, O, l1, l2);
    V(l, D, l3, l4);
    V(r, O, r1, r2);
    V(r, D, r3, r4);
    double cl = C2(l1, l2, l3, l4), cr = C2(r1, r2, r3, r4);
    bool BL = cl > eps, BR = cr > eps;
    if (!BL && !BR) break;
    if (BL) {
      double dl = Dot(l1, l2, l3, l4);
      cot_L = dl / cl;
      do {
        next = Next(El, O);
        V(Other(next, O), O, u1, v1);
        V(Other(next, O), D, u2, v2);
        n1 = C2(u1, v1, u2, v2);
        if (!(n1 > eps)) break;
        cot_n = Dot(u1, v1, u2, v2) / n1;
        if (cot_n > cot_L) break;
        Remove(El);
        El = next;
        cot_L = cot_n;
      } while (1);
    }
    if (BR) {
      double dr = Dot(r1, r2, r3, r4);
      cot_R = dr / cr;
      do {
        prev = Prev(Er, D);
        V(Other(prev, D), O, u1, v1);
        V(Other(prev, D), D, u2, v2);
        P1 = C2(u1, v1, u2, v2);
        if (!(P1 > eps)) break;
        cot_P = Dot(u1, v1, u2, v2) / P1;
        if (cot_P > cot_R) break;
        Remove(Er);
        Er = prev;
        cot_R = cot_P;
      } while (1);
    }
    l = Other(El, O);
    r = Other(Er, D);
    if (!BL || (BL && BR && cot_R < cot_L)) {
      B = Join(B, O, Er, r, 0);
      D = r;
    } else {
      B = Join(El, l, B, D, 0);
      O = l;
    }
  } while (1);
}

void Divide(int s, int t, edge **L, edge **R) {
  edge *a, *b, *c, *ll, *lr, *rl, *rr, *tangent;
  int n = t - s + 1;
  if (n == 2) *L = *R = Make_edge(Q[s], Q[t]);
  else if (n == 3) {
    a = Make_edge(Q[s], Q[s + 1]), b = Make_edge(Q[s + 1], Q[t]);
    Splice(a, b, Q[s + 1]);
    double v = C3(Q[s], Q[s + 1], Q[t]);
    if (v > eps) {
      c = Join(a, Q[s], b, Q[t], 0);
      *L = a;
      *R = b;
    } else if (v < -eps) {
      c = Join(a, Q[s], b, Q[t], 1);
      *L = c;
      *R = c;
    } else {
      *L = a;
      *R = b;
    }
  } else if (n > 3) {
    int split = (s + t) / 2;
    Divide(s, split, &ll, &lr);
    Divide(split + 1, t, &rl, &rr);
    Merge(lr, Q[split], rl, Q[split + 1], &tangent);
    if (Oi(tangent) == Q[s])
      ll = tangent;
    if (Dt(tangent) == Q[t])
      rr = tangent;
    *L = ll;
    *R = rr;
  }
}
void Make_Graph() {
  edge *start, *e;
  point *u, *v;
  int i;
  for (i = 0; i < n; i++) {
    u = &p[i];
    start = e = u->in;
    do {
      v = Other(e, u);
      if (u < v) {
        E[M].u = u - p, E[M].v = v - p;
        E[M++].w = dis(u, v);
        if (M >= aix * maxn) OLE();
      }
      e = Next(e, u);
    } while (e != start);
  }
}
void solve() {
  int i, test;
  scanf("%d", &test);
  while (test) {
    test--;
    n = 0;
    double ans = -1;
    scanf("%d", &n);
    for (i = 0; i < n; i++) {
      scanf("%lf%lf", &p[i].x, &p[i].y);
      p[i].index = i;
      p[i].in = NULL;
    }
    Alloc_memory();
    if (n == 1 || n == 0) {
      continue;
    }   // else RE
    sort(p, p + n);

//=========点不能有重点,有的话不满足voronoi图的性质了 
    for (i = 0; i < n; i++)
      Q[i] = p + i;
    edge *L, *R;
    Divide(0, n - 1, &L, &R);
    M = 0;
    Make_Graph();
    Kruskal();
//    puts("---------------------");
  }
}

int main() {
  freopen("input.txt", "r", stdin);
  freopen("output.txt", "w", stdout);
  solve();
  return 0;
}


\end{lstlisting}