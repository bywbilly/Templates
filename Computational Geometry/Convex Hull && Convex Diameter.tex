\begin{lstlisting}

inline bool turnLeft(const point &a, const point &b, const point &c) {
  return sign(det(b - a, c - a)) >= 0;
}

inline bool turnRight(const point &a, const point &b, const point &c) {
  return sign(det(b - a, c - a)) <= 0;
}

inline bool cmpByXY(const point &a, const point &b) {
  int c = sign(a.x - b.x);
  if (c != 0) return c < 0;
  return sign(a.y - b.y) < 0;
}

vector<point> convexHull(vector<point> &a) {
  int n = (int)a.size(), cnt = 0;
  sort(a.begin(), a.end(), cmpByXY);
  vector<point> ret;
  ret.reserve(n * 2);
  for (int i = 0; i < n; ++i) {
    while (cnt > 1 && turnLeft(ret[cnt - 2], a[i], ret[cnt - 1])) {
      --cnt;
      ret.pop_back();
    }
    ret.push_back(a[i]);
    ++cnt;
  }
  int fixed = cnt;
  for (int i = n - 2; i >= 0; --i) { // n - 1 must be in stack
    while (cnt > fixed && turnLeft(ret[cnt - 2], a[i], ret[cnt - 1])) {
      --cnt;
      ret.pop_back();
    }
    ret.push_back(a[i]);
    ++cnt;
  }
  // the lowest point will occur twice, i.e. ret.front() == ret.back()
  return ret;
}

double convexDiameter(const vector<point> &ps) {
  int n = ps.size();
  if (n < 2) return 0;
  if (n == 2) return (ps[1] - ps[0]).len();
  int nx, ny, y = 1;
  double k;
  double ans = 0;
  for (int x = 0; x < n; ++x) {
    nx = x + 1;
    for ( ; ; y = ny) {
      ny = y == n - 1 ? 0 : y + 1;
      k = det(ps[nx] - ps[x], ps[ny] - ps[y]);
      if (k <= 0) break;
    }
    ans = max(ans, (ps[x] - ps[y]).len());
    if (sign(k) == 0)
      ans = max(ans, (ps[x] - ps[ny]).len());
  }
  return ans;
}

\end{lstlisting}