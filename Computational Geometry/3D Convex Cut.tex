\begin{lstlisting}

vector<vector<point> > convexCut(const vector<vector<point> > &pss, const point &p, const point &o) {
  vector<vector<point> > res;
  vector<point> sec;
  for (unsigned itr = 0, size = pss.size(); itr < size; ++itr) {
    const vector<point> &ps = pss[itr];
    int n = ps.size();
    vector<point> qs;
    bool dif = false;
    for (int i = 0; i < n; ++i) {
      int d1 = sign( dot(o, ps[i] - p) );
      int d2 = sign( dot(o, ps[(i + 1) % n] - p) );
      if (d1 <= 0) qs.push_back(ps[i]);
      if (d1 * d2 < 0) {
        point q;
        isFL(p, o, ps[i], ps[(i + 1) % n], q); // must return true
        qs.push_back(q);
        sec.push_back(q);
      }
      if (d1 == 0) sec.push_back(ps[i]);
      else dif = true;
      dif |= dot(o, det(ps[(i + 1) % n] - ps[i], ps[(i + 2) % n] - ps[i])) < -EPS;
    }
    if (!qs.empty() && dif)
      res.insert(res.end(), qs.begin(), qs.end());
  }
  if (!sec.empty()) {
    vector<point> tmp( convexHull2D(sec, o) );
    res.insert(res.end(), tmp.begin(), tmp.end());
  }
  return res;
}

vector<vector<point> > initConvex() {
  vector<vector<point> > pss(6, vector<point>(4));
  pss[0][0] = pss[1][0] = pss[2][0] = point(-INF, -INF, -INF);
  pss[0][3] = pss[1][1] = pss[5][2] = point(-INF, -INF,  INF);
  pss[0][1] = pss[2][3] = pss[4][2] = point(-INF,  INF, -INF);
  pss[0][2] = pss[5][3] = pss[4][1] = point(-INF,  INF,  INF);
  pss[1][3] = pss[2][1] = pss[3][2] = point( INF, -INF, -INF);
  pss[1][2] = pss[5][1] = pss[3][3] = point( INF, -INF,  INF);
  pss[2][2] = pss[4][3] = pss[3][1] = point( INF,  INF, -INF);
  pss[5][0] = pss[4][0] = pss[3][0] = point( INF,  INF,  INF);
  return pss;
}

\end{lstlisting}