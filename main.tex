\documentclass[10pt, a4paper]{article}
\usepackage{listings}
\usepackage{geometry}

\usepackage{ctex, amsmath, amsthm, graphicx, amsfonts, fancyhdr, ctable, pict2e, CJKulem}
\usepackage[CJKbookmarks, colorlinks = true, linkcolor = red, anchorcolor = blue, citecolor = blue, pdfstartview={FitH}]{hyperref}

%\geometry{left=2.7cm, right=2.7cm, top=3cm, bottom=3cm}

\geometry{left=2.0cm, right=2.0cm, top=2.5cm, bottom=2.5cm}

\lstset{language=C++, breaklines,numbers=left,frame=leftline,
    escapeinside=``,extendedchars=false,
    xleftmargin=0em,xrightmargin=0em,
    basicstyle=\ttfamily\small, tabsize=2,
}

\begin{document}

\title{Call It Magic - Standard Code Library}
\date{}
\tableofcontents
\newpage

\section{计算几何}

  \subsection{二维计算几何基本操作}
    \begin{lstlisting}

const double EPS = 1e-7;
const double PI = 3.1415926535897932384626;
const int INF = 1000000000;

int sign(const double &a, const double &eps = EPS) {
  return a < -eps ? -1 : int(a > eps);
}

double sqr(const double &x) {
  return x * x;
}

double Sqrt(const double &x) {
  return x < 0 ? 0 : sqrt(x);
}

double arcSin(const double &a) {
  if (a <= -1.0) return -PI / 2;
  if (a >=  1.0) return  PI / 2;
  return asin(a);
}

double arcCos(const double &a) {
  if (a <= -1.0) return PI;
  if (a >=  1.0) return 0;
  return acos(a);
}

struct point {
  double x, y;

  point() : x(0.0), y(0.0) {
  }

  point(const double &x, const double &y) : x(x), y(y) {
  }

  point operator +(const point &rhs) const {
    return point(x + rhs.x, y + rhs.y);
  }

  point operator -(const point &rhs) const {
    return point(x - rhs.x, y - rhs.y);
  }

  point operator *(const double &k) const {
    return point(x * k, y * k);
  }

  point operator /(const double &k) const {
    return point(x / k, y / k);
  }

  double len() const {
    return hypot(x, y);
  }

  double norm() const {
    return x * x + y * y;
  }

  point unit() const {
    double k = len();
    return point(x / k, y / k);
  }

  point rot(const double &a) const { // counter-clockwise
    return point(x * cos(a) - y * sin(a), x * sin(a) + y * cos(a));
  }

  point rot90() const { // counter-clockwise
    return point(-y, x);
  }

  point rot180() const { // counter-clockwise
    return point(-x, -y);
  }

  point rot270() const { // counter-clockwise
    return point(y, -x);
  }

  point project(const point &p1, const point &p2) const {
    const point &q = *this;
    return p1 + (p2 - p1) * (dot(p2 - p1, q - p1) / (p2 - p1).norm());
  }

  friend double dot(const point &a, const point &b) {
    return a.x * b.x + a.y * b.y;
  }

  friend double det(const point &a, const point &b) {
    return a.x * b.y - a.y * b.x;
  }

  bool onSeg(const point &a, const point &b) const { // [a, b] inclusive
    const point &c = *this;
    return sign(dot(a - c, b - c)) <= 0 && sign(det(b - a, c - a)) == 0;
  }

  double distLP(const point &p1, const point &p2) const { // dist from *this to line p1->p2
    const point &q = *this;
    return fabs(det(p2 - p1, q - p1)) / (p2 - p1).len();
  }

  double distSP(const point &p1, const point &p2) const { // dist from *this to segment [p1, p2]
    const point &q = *this;
    if (dot(p2 - p1, q - p1) < EPS) return (q - p1).len();
    if (dot(p1 - p2, q - p2) < EPS) return (q - p2).len();
    return distLP(p1, p2);
  }

};

bool lineIntersect(const point &a, const point &b, const point &c, const point &d, point &e) {
  double s1 = det(c - a, d - a);
  double s2 = det(d - b, c - b);
  if (!sign(s1 + s2)) return 0;
  e = (b - a) * (s1 / (s1 + s2)) + a;
  return 1;
}

bool segIntersect(const point &a, const point &b, const point &c, const point &d, point &e) { // A simpler version
  double s1 = det(c - a, d - a), s2 = det(d - b, c - b);
  if (!sign(s1 + s2)) return 0;
  e = (b - a) * (s1 / (s1 + s2)) + a;
  return e.onSeg(a, b) && e.onSeg(c, d);
}

// returns 0 if not intersect, 1 if proper intersect, 2 if improper intersect
int iCnt;
int segIntersectCheck(const point &a, const point &b, const point &c, const point &d, point &o) {
  static double s1, s2, s3, s4;
  int d1 = sign(s1 = det(b - a, c - a)), d2 = sign(s2 = det(b - a, d - a));
  int d3 = sign(s3 = det(d - c, a - c)), d4 = sign(s4 = det(d - c, b - c));
  if ((d1 ^ d2) == -2 && (d3 ^ d4) == -2) {
    o = (c * s2 - d * s1) / (s2 - s1);
    return 1;
  }
  iCnt = 0;
  if (d1 == 0 && c.onSeg(a, b)) o = c, ++iCnt;
  if (d2 == 0 && d.onSeg(a, b)) o = d, ++iCnt;
  if (d3 == 0 && a.onSeg(c, d)) o = a, ++iCnt;
  if (d4 == 0 && b.onSeg(c, d)) o = b, ++iCnt;
  return iCnt ? 2 : 0;
}

struct circle {
  point o;
  double r, rSqure; // r ^ 2

  circle() :
      o(), r(0.0), rSqure(0.0) {
  }

  circle(const point &o, const double &r) :
      o(o), r(r), rSqure(r * r) {
  }

  bool inside(const point &a) { // not strict
    return (a - o).len() < r + EPS;
  }

  bool contain(const circle &b) const { // not strict
    return sign(b.r + (o - b.o).len() - r) <= 0;
  }

  bool disjunct(const circle &b) const { // not strict
    return sign(b.r + r - (o - b.o).len()) <= 0;
  }

  int isCL(const point &p1, const point &p2, point &a, point &b) const {
    double x = dot(p1 - o, p2 - p1);
    double y = (p2 - p1).norm();
    double d = x * x - y * ((p1 - o).norm() - rSqure);
    if (d < -EPS)
      return 0;
    if (d < 0)
      d = 0;
    point q1 = p1 - (p2 - p1) * (x / y);
    point q2 = (p2 - p1) * (sqrt(d) / y);
    a = q1 - q2;
    b = q1 + q2;
    return q2.len() < EPS ? 1 : 2;
  }

  int tanCP(const point &p, point &a, point &b) const {
    double x = (p - o).norm();
    double d = x - rSqure;
    if (d < -EPS) return 0;
    if (d < 0) d = 0;
    point q1 = (p - o) * (rSqure / x);
    point q2 = ((p - o) * (-r * sqrt(d) / x)).rot90();
    a = o + (q1 - q2);
    b = o + (q1 + q2);
    return q2.len() < EPS ? 1 : 2;
  }
};

bool checkCrossCS(const circle &cir, const point &p1, const point &p2) { // not strict
  const point &c = cir.o;
  const double &r = cir.r;
  return c.distSP(p1, p2) < r + EPS
      && (r < (c - p1).len() + EPS || r < (c - p2).len() + EPS);
}

bool checkCrossCC(const circle &cir1, const circle &cir2) { // not strict
  const double &r1 = cir1.r;
  const double &r2 = cir2.r;
  double d = (cir1.o - cir2.o).len();
  return d < r1 + r2 + EPS && fabs(r1 - r2) < d + EPS;
}

int isCC(const circle &cir1, const circle &cir2, point &a, point &b) {
  const point &c1 = cir1.o, &c2 = cir2.o;
  double x = (c1 - c2).norm();
  double y = ((cir1.rSqure - cir2.rSqure) / x + 1) / 2;
  double d = cir1.rSqure / x - y * y;
  if (d < -EPS) return 0;
  if (d < 0) d = 0;
  point q1 = c1 + (c2 - c1) * y;
  point q2 = ((c2 - c1) * sqrt(d)).rot90();
  a = q1 - q2;
  b = q1 + q2;
  return q2.len() < EPS ? 1 : 2;
}

vector<pair<point, point> > tanCC(const circle &cir1, const circle &cir2) {
  vector<pair<point, point> > list;
  const point &c1 = cir1.o, &c2 = cir2.o;
  double r1 = cir1.r, r2 = cir2.r;
  point p, a1, b1, a2, b2;
  int s1, s2;
  if (sign(r1 - r2) == 0) {
    p = c2 - c1;
    p = (p * (r1 / p.len())).rot90();
    list.push_back(make_pair(c1 + p, c2 + p));
    list.push_back(make_pair(c1 - p, c2 - p));
  } else {
    p = (c2 * r1 - c1 * r2) / (r1 - r2);
    s1 = cir1.tanCP(p, a1, b1);
    s2 = cir2.tanCP(p, a2, b2);
    if (s1 >= 1 && s2 >= 1) {
      list.push_back(make_pair(a1, a2));
      list.push_back(make_pair(b1, b2));
    }
  }
  p = (c1 * r2 + c2 * r1) / (r1 + r2);
  s1 = cir1.tanCP(p, a1, b1);
  s2 = cir2.tanCP(p, a2, b2);
  if (s1 >= 1 && s2 >= 1) {
    list.push_back(make_pair(a1, a2));
    list.push_back(make_pair(b1, b2));
  }
  return list;
}

double areaCT(const circle &cir, point pa, point pb) {
  pa = pa - cir.o;
  pb = pb - cir.o;
  double R = cir.r;
  if (pa.len() < pb.len()) swap(pa, pb);
  if (pb.len() < EPS) return 0;
  point pc = pb - pa;
  double a = pa.len(), b = pb.len(), c = pc.len();
  double cosB = dot(pb, pc) / b / c, B = acos(cosB);
  double cosC = dot(pa, pb) / a / b, C = acos(cosC);
  double S, h, theta;
  if (b > R) {
    S = C * 0.5 * R * R;
    h = b * a * sin(C) / c;
    if (h < R && B < PI * 0.5)
      S -= acos(h / R) * R * R - h * sqrt(R * R - h * h);
  } else if (a > R) {
    theta = PI - B - asin(sin(B) / R * b);
    S = 0.5 * b * R * sin(theta) + (C - theta) * 0.5 * R * R;
  } else {
    S = 0.5 * sin(C) * b * a;
  }
  return S;
}

circle minCircle(const point &a, const point &b) {
  return circle((a + b) * 0.5, (b - a).len() * 0.5);
}

circle minCircle(const point &a, const point &b, const point &c) { // WARNING: Obtuse triangle is not considered
  double A = 2 * a.x - 2 * b.x;
  double B = 2 * a.y - 2 * b.y;
  double C = a.x * a.x + a.y * a.y - b.x * b.x - b.y * b.y;
  double D = 2 * a.x - 2 * c.x;
  double E = 2 * a.y - 2 * c.y;
  double F = a.x * a.x + a.y * a.y - c.x * c.x - c.y * c.y;
  point p((C * E - B * F) / (A * E - B * D), (A * F - C * D) / (A * E - B * D));
  return circle(p, (p - a).len());
}

circle minCircle(point P[], int N) {
  if (N == 1) return circle(P[1], 0.0);
  random_shuffle(P + 1, P + N + 1);
  circle O = minCircle(P[1], P[2]);
  Rep(i, 3, N) if(!O.inside(P[i])) {
    O = minCircle(P[1], P[i]);
    Foru(j, 2, i) if(!O.inside(P[j])) {
      O = minCircle(P[i], P[j]);
      Foru(k, 1, j) if(!O.inside(P[k]))
        O = minCircle(P[i], P[j], P[k]);
    }
  }
  return O;
}

\end{lstlisting}
    
  \subsection{圆的面积模板}
    \begin{lstlisting}
struct Event {
  point p;
  double alpha;
  int add;
  Event(): p(), alpha(0.0), add(0) {}
  Event(const point &p, const double &alpha, const int &add): p(p), alpha(alpha), add(add) {}
  bool operator < (const Event &other) const {
    return alpha < other.alpha;
  }
};

void circleKCover(circle *c, int N, double *area) { // area[k] : covered not less than k times
  static bool overlap[MAXN][MAXN], g[MAXN][MAXN];
  Rep(i, 0, N + 1) area[i] = 0.0;
  Rep(i, 1, N) Rep(j, 1, N) overlap[i][j] = c[i].contain(c[j]);
  Rep(i, 1, N) Rep(j, 1, N) g[i][j] = !(overlap[i][j] || overlap[j][i] || c[i].disjunct(c[j]));
  Rep(i, 1, N) {
    static Event events[MAXN * 2 + 1];
    int totE = 0, cnt = 1;
    Rep(j, 1, N) if (j != i && overlap[j][i]) ++cnt;
    Rep(j, 1, N) if (j != i && g[i][j]) {
      circle &a = c[i], &b = c[j];
      double l = (a.o - b.o).norm();
      double s = ((a.r - b.r) * (a.r + b.r) / l + 1) * 0.5;
      double t = sqrt(-(l - sqr(a.r - b.r)) * (l - sqr(a.r + b.r)) / (l * l * 4.0));
      point dir = b.o - a.o;
      point Ndir = point(-dir.y, dir.x);
      point aa = a.o + dir * s + Ndir * t;
      point bb = a.o + dir * s - Ndir * t;
      double A = atan2(aa.y - a.o.y, aa.x - a.o.x);
      double B = atan2(bb.y - a.o.y, bb.x - a.o.x);
      events[totE++] = Event(bb, B, 1);
      events[totE++] = Event(aa, A, -1);
      if (B > A) ++cnt;
    }
    if (totE == 0) {
      area[cnt] += PI * c[i].r2;
      continue;
    }
    sort(events, events + totE);
    events[totE] = events[0];
    Foru(j, 0, totE) {
      cnt += events[j].add;
      area[cnt] += 0.5 * det(events[j].p, events[j + 1].p);
      double theta = events[j + 1].alpha - events[j].alpha;
      if (theta < 0) theta += 2.0 * PI;
      area[cnt] += 0.5 * c[i].r2 * (theta - sin(theta));
    }
  }
}

\end{lstlisting}

  \subsection{多边形相关}
    \begin{lstlisting}

#define MAXN 1033
struct Polygon { // stored in [0, n)
  int n;
  point list[MAXN];

  Polygon cut(const point &a, const point &b) {
    static Polygon res;
    static point o;
    res.n = 0;
    for (int i = 0; i < n; ++i) {
      int s1 = sign(det(list[i] - a, b - a));
      int s2 = sign(det(list[(i + 1) % n] - a, b - a));
      if (s1 <= 0) res.list[res.n++] = list[i];
      if (s1 * s2 < 0) {
        lineIntersect(a, b, list[i], list[(i + 1) % n], o);
        res.list[res.n++] = o;
      }
    }
    return res;
  }

  bool contain(const point &p) const { // 1 if on border or inner, 0 if outter
    static point A, B;
    int res = 0;
    for (int i = 0; i < n; ++i) {
      A = list[i];
      B = list[(i + 1) % n];
      if (p.onSeg(A, B)) return 1;
      if (sign(A.y - B.y) <= 0) swap(A, B);
      if (sign(p.y - A.y) > 0) continue;
      if (sign(p.y - B.y) <= 0) continue;
      res += (int)(sign(det(B - p, A - p)) > 0);
    }
    return res & 1;
  }
};

\end{lstlisting}

  \subsection{最大面积空凸包}
    \begin{lstlisting}
/**
 * Time Complexity: O(N ^ 3)
 */


inline bool toUpRight(const point &a, const point &b) {
  int c = sign(b.y - a.y);
  if (c > 0) return true;
  return c == 0 && sign(b.x - a.x) > 0;
}

inline bool cmpByPolarAngle(const point &a, const point &b) { // counter-clockwise, shorter first if they share the same polar angle
  int c = sign(det(a, b));
  if (c != 0) return c > 0;
  return sign(b.len() - a.len()) > 0;
}

double maxEmptyConvexHull(point p[], int N) {
  static double dp[133][133];
  static point vec[133];
  static int seq[133]; // empty triangles formed with (0, 0), vec[o], vec[ seq[i] ]

  double ans = 0.0;
  Rep(o, 1, N) {
    int totVec = 0;
    Rep(i, 1, N) if (toUpRight(p[o], p[i])) 
      vec[++totVec] = p[i] - p[o];
    sort(vec + 1, vec + totVec + 1, cmpByPolarAngle);
    Rep(i, 1, totVec) Rep(j, 1, totVec) dp[i][j] = 0.0;
    
    Rep(k, 2, totVec) {
      int i = k - 1;
      while (i > 0 && sign( det(vec[k], vec[i]) ) == 0) --i;
      
      int totSeq = 0;
      for (int j; i > 0; i = j) {
        seq[++totSeq] = i;
        for (j = i - 1; j > 0 && sign(det(vec[i] - vec[k], vec[j] - vec[k])) > 0; --j);

        double v = det(vec[i], vec[k]) * 0.5;
        if (j > 0) v += dp[i][j];
        dp[k][i] = v;
        Up(ans, v);
      }
      for (int i = totSeq - 1; i >= 1; --i)
        Up( dp[k][ seq[i] ], dp[k][seq[i + 1]] );
    }
  }
  return ans;
}

\end{lstlisting}

  \subsection{最近点对}
    \begin{lstlisting}

int N;
point p[maxn];

bool cmpByX(const point &a, const point &b) {
  return sign(a.x - b.x) < 0; 
}

bool cmpByY(const int &a, const int &b) {
  return p[a].y < p[b].y; 
}

double minimalDistance(point *c, int n, int *ys) {
  double ret = 1e+20;
  if (n < 20) {
    Foru(i, 0, n) Foru(j, i + 1, n)
      Down(ret, (c[i] - c[j]).len() );
    sort(ys, ys + n, cmpByY);
    return ret; 
  }
  static int mergeTo[maxn];
  int mid = n / 2;
  double xmid = c[mid].x;
  ret = min(minimalDistance(c, mid, ys), minimalDistance(c + mid, n - mid, ys + mid));
  merge(ys, ys + mid, ys + mid, ys + n, mergeTo, cmpByY);
  copy(mergeTo, mergeTo + n, ys);

  Foru(i, 0, n) {
    while (i < n && sign(fabs(p[ys[i]].x - xmid) - ret) > 0) ++i;
    int cnt = 0;
    Foru(j, i + 1, n)
      if (sign(p[ys[j]].y - p[ys[i]].y - ret) > 0) break;
      else if (sign(fabs(p[ys[j]].x - xmid) - ret) <= 0) {
        ret = min(ret, (p[ys[i]] - p[ys[j]]).len());
        if (++cnt >= 10) break;
      }
  }
  return ret; 
}

void work() {
  sort(p, p + n, cmpByX);
  Foru(i, 0, n) ys[i] = i;
  double ans = minimalDistance(p, n, ys);
}

\end{lstlisting}

  \subsection{凸包与点集直径}
    \begin{lstlisting}

inline bool turnLeft(const point &a, const point &b, const point &c) {
  return sign(det(b - a, c - a)) >= 0;
}

inline bool turnRight(const point &a, const point &b, const point &c) {
  return sign(det(b - a, c - a)) <= 0;
}

inline bool cmpByXY(const point &a, const point &b) {
  int c = sign(a.x - b.x);
  if (c != 0) return c < 0;
  return sign(a.y - b.y) < 0;
}

vector<point> convexHull(vector<point> &a) {
  int n = (int)a.size(), cnt = 0;
  sort(a.begin(), a.end(), cmpByXY);
  vector<point> ret;
  ret.reserve(n * 2);
  for (int i = 0; i < n; ++i) {
    while (cnt > 1 && turnLeft(ret[cnt - 2], a[i], ret[cnt - 1])) {
      --cnt;
      ret.pop_back();
    }
    ret.push_back(a[i]);
    ++cnt;
  }
  int fixed = cnt;
  for (int i = n - 2; i >= 0; --i) { // n - 1 must be in stack
    while (cnt > fixed && turnLeft(ret[cnt - 2], a[i], ret[cnt - 1])) {
      --cnt;
      ret.pop_back();
    }
    ret.push_back(a[i]);
    ++cnt;
  }
  // the lowest point will occur twice, i.e. ret.front() == ret.back()
  return ret;
}

double convexDiameter(const vector<point> &ps) {
  int n = ps.size();
  if (n < 2) return 0;
  if (n == 2) return (ps[1] - ps[0]).len();
  int nx, ny, y = 1;
  double k;
  double ans = 0;
  for (int x = 0; x < n; ++x) {
    nx = x + 1;
    for ( ; ; y = ny) {
      ny = y == n - 1 ? 0 : y + 1;
      k = det(ps[nx] - ps[x], ps[ny] - ps[y]);
      if (k <= 0) break;
    }
    ans = max(ans, (ps[x] - ps[y]).len());
    if (sign(k) == 0)
      ans = max(ans, (ps[x] - ps[ny]).len());
  }
  return ans;
}

\end{lstlisting}

  \subsection{三角形的五心}
    \begin{itemize}
    \item
      重心 $\overrightarrow{G} = \frac{\overrightarrow{A} + \overrightarrow{B} + \overrightarrow{C}}{3}$
    \item
      内心 $\overrightarrow{I} = \frac{a\overrightarrow{A} + b\overrightarrow{B} + c\overrightarrow{C}}{a + b + c}$
    \item
      外心 $\overrightarrow{O} = \frac{\overrightarrow{A} + \overrightarrow{B} + \frac{\overrightarrow{AC} \cdot \overrightarrow{BC}}{\overrightarrow{AB} \times \overrightarrow{BC}} \overrightarrow{AB}^{T}}{2}, R = \frac{abc}{4S}$
    \item
      垂心 $\overrightarrow{H} = 3\overrightarrow{G} - 2\overrightarrow{O} = \overrightarrow{C} + \frac{\overrightarrow{BC} \cdot \overrightarrow{AC}}{\overrightarrow{BC} \times \overrightarrow{AC}} \overrightarrow{AB} ^ {T}$
    \item
      旁心(三个) $\frac{-a\overrightarrow{A} + b\overrightarrow{B} + c\overrightarrow{C}}{-a + b + c}$
    \end{itemize}

  \subsection{Farmland}
    \begin{lstlisting}

struct node {
  int begin[MAXN], *end;
} a[MAXN];

int n, m, S;
bool visit[MAXN][MAXN];
point p[MAXN];

inline bool cmp(const int &a, const int &b) {
  return sign(atan2(p[a].y - p[S].y, p[a].x - p[S].x)
        - atan2(p[b].y - p[S].y, p[b].x - p[S].x)) < 0;
}

void init() {
  scanf("%d", &n);
  Rep(i, 1, n) Rep(j, 1, n) visit[i][j] = 0;
  Rep(i, 1, n) a[i].end = a[i].begin;
  Rep(i, 1, n) {
    int d;
    scanf("%d", &d);
    scanf("%lf%lf", &p[i].x, &p[i].y);
    scanf("%d", &d);
    Rep(j, 1, d) scanf("%d", a[i].end++);
  }
  scanf("%d", &m);
  for (S = 1; S <= n; ++S) sort(a[S].begin, a[S].end, cmp);
}

bool check(int b1, int b2) {
  static pii l[MAXN * 2 + 1];
  static bool valid[MAXN];
  double area = 0;
  int tp = 0, *k;
  l[0] = pii(b1, b2);
  for ( ; ; ) {
    int p1 = l[tp].first, p2 = l[tp].second;
    visit[p1][p2] = true;
    area += det(p[p1], p[p2]);
    for (k = a[p2].begin; k != a[p2].end; ++k) if (*k == p1) break;
    k = (k == a[p2].begin) ? (a[p2].end - 1) : (k - 1);
    l[++tp] = pii(p2, *k);
    if (l[tp] == l[0]) break;
  }
  if (sign(area) < 0 || tp < 3 || tp != m) return 0;
  Rep(i, 1, n) valid[i] = false;
  for (int i = 0; i < tp; i++) {
    int p = l[i].first;
    if (valid[p]) return 0;
    valid[p] = 1;
  }
  return 1;    
}

void work() {
  int ans = 0;
  Rep(x, 1, n) {
    for (int *itr = a[x].begin; itr != a[x].end; ++itr) {
      int y = *itr;
      if (visit[x][y]) continue;
      if (check(x, y)) ++ans;
    }
  }
  printf("%d\n", ans);      
}

int main() {
  int T;
  scanf("%d", &T);
  while (T--) {
    init();
    work();
  }
  return 0;
}


\end{lstlisting}

  \subsection{Voronoi图}
    \begin{lstlisting}

#define Oi(e) ((e)->oi)
#define Dt(e) ((e)->dt)
#define On(e) ((e)->on)
#define Op(e) ((e)->op)
#define Dn(e) ((e)->dn)
#define Dp(e) ((e)->dp)
#define Other(e, p) ((e)->oi == p ? (e)->dt : (e)->oi)
#define Next(e, p) ((e)->oi == p ? (e)->on : (e)->dn)
#define Prev(e, p) ((e)->oi == p ? (e)->op : (e)->dp)
#define V(p1, p2, u, v) (u = p2->x - p1->x, v = p2->y - p1->y)
#define C2(u1, v1, u2, v2) (u1 * v2 - v1 * u2)
#define C3(p1, p2, p3) ((p2->x - p1->x) * (p3->y - p1->y) - (p2->y - p1->y) * (p3->x - p1->x))
#define Dot(u1, v1, u2, v2) (u1 * u2 + v1 * v2)
#define dis(a,b) (sqrt( (a->x - b->x) * (a->x - b->x) + (a->y - b->y) * (a->y - b->y) ))

const int maxn = 110024;
const double eps = 1e-7;
const int aix = 4;
int n, M, k;

struct gEdge {
  int u, v;
  double w;
  bool operator <(const gEdge &e1) const {
    return w < e1.w - eps;
  }
} E[aix * maxn], MST[maxn];

int b[maxn];
int Find(int x) {
  while (x != b[x]) {
    b[x] = b[b[x]];
    x = b[x];
  }
  return x;
}

void Kruskal() {
  int m1, m2;
  memset(b, 0, sizeof(b));
  for (int i = 0; i < n; i++) b[i] = i;
  sort(E, E + M);
  for (int i = 0, kk = 0; i < M && kk < n - 1; i++) {
    m1 = Find(E[i].u);
    m2 = Find(E[i].v);
    if (m1 != m2) {
      b[m1] = m2;
      MST[kk++] = E[i];
    }
  }
}

struct point {
  double x, y;
  int index;
  struct edge *in;
  bool operator <(const point &p1) const {
    return x < p1.x - eps || (abs(x - p1.x) <= eps && y < p1.y - eps);
  }
};

struct edge {
  point *oi, *dt;
  edge *on, *op, *dn, *dp;
};

point p[maxn], *Q[maxn];
edge mem[aix * maxn], *elist[aix * maxn];
int nfree;

//memory
void Alloc_memory() {
  nfree = aix * n;
  edge *e = mem;
  for (int i = 0; i < nfree; i++)
    elist[i] = e++;
}

//Add an edge to a ring of edges
void Splice(edge *a, edge *b, point *v) {
  edge *next;
  if (Oi(a) == v)
    next = On(a), On(a) = b;
  else
    next = Dn(a), Dn(a) = b;
  if (Oi(next) == v)
    Op(next) = b;
  else
    Dp(next) = b;
  if (Oi(b) == v)
    On(b) = next, Op(b) = a;
  else
    Dn(b) = next, Dp(b) = a;
}

//Initialise a new edge
edge *Make_edge(point *u, point *v) {
  edge *e = elist[--nfree];
  e->on = e->op = e->dn = e->dp = e;
  e->oi = u;
  e->dt = v;
  if (!u->in) u->in = e;
  if (!v->in) v->in = e;
  return e;
}

//Creates a new edge and adds it to two rings of edges.
edge *Join(edge *a, point *u, edge *b, point *v, int side) {
  edge *e = Make_edge(u, v);
  if (side == 1) {
    if (Oi(a) == u)
      Splice(Op(a), e, u);
    else
      Splice(Dp(a), e, u);
    Splice(b, e, v);
  } else {
    Splice(a, e, u);
    if (Oi(b) == v)
      Splice(Op(b), e, v);
    else
      Splice(Dp(b), e, v);
  }
  return e;
}

//Remove an edge
void Remove(edge *e) {
  point *u = Oi(e), *v = Dt(e);
  if (u->in == e)
    u->in = e->on;
  if (v->in == e)
    v->in = e->dn;
  if (Oi(e->on) == u)
    e->on->op = e->op;
  else
    e->on->dp = e->op;
  if (Oi(e->op) == u)
    e->op->on = e->on;
  else
    e->op->dn = e->on;
  if (Oi(e->dn) == v)
    e->dn->op = e->dp;
  else
    e->dn->dp = e->dp;
  if (Oi(e->dp) == v)
    e->dp->on = e->dn;
  else
    e->dp->dn = e->dn;
  elist[nfree++] = e;
}

//Determines the lower tangent of two triangulations
void Low_tangent(edge *e_l, point *o_l, edge *e_r, point *o_r, edge **l_low, point **OL, edge **r_low, point **OR) {
  point *d_l = Other(e_l, o_l), *d_r = Other(e_r, o_r);
  while (1) {
    if (C3(o_l, o_r, d_l) < -eps) {
      e_l = Prev(e_l, d_l);
      o_l = d_l;
      d_l = Other(e_l, o_l);
    } else if (C3(o_l, o_r, d_r) < -eps) {
      e_r = Next(e_r, d_r);
      o_r = d_r;
      d_r = Other(e_r, o_r);
    } else
      break;
  }
  *OL = o_l, *OR = o_r;
  *l_low = e_l, *r_low = e_r;
}

void Merge(edge *lr, point *s, edge *rl, point *u, edge **tangent) {
  double l1, l2, l3, l4, r1, r2, r3, r4, cot_L, cot_R, u1, v1, u2, v2, n1, cot_n, P1, cot_P;
  point *O, *D, *OR, *OL;
  edge *B, *L, *R;
  Low_tangent(lr, s, rl, u, &L, &OL, &R, &OR);
  *tangent = B = Join(L, OL, R, OR, 0);
  O = OL, D = OR;
  do {
    edge *El = Next(B, O), *Er = Prev(B, D), *next, *prev;
    point *l = Other(El, O), *r = Other(Er, D);
    V(l, O, l1, l2);
    V(l, D, l3, l4);
    V(r, O, r1, r2);
    V(r, D, r3, r4);
    double cl = C2(l1, l2, l3, l4), cr = C2(r1, r2, r3, r4);
    bool BL = cl > eps, BR = cr > eps;
    if (!BL && !BR) break;
    if (BL) {
      double dl = Dot(l1, l2, l3, l4);
      cot_L = dl / cl;
      do {
        next = Next(El, O);
        V(Other(next, O), O, u1, v1);
        V(Other(next, O), D, u2, v2);
        n1 = C2(u1, v1, u2, v2);
        if (!(n1 > eps)) break;
        cot_n = Dot(u1, v1, u2, v2) / n1;
        if (cot_n > cot_L) break;
        Remove(El);
        El = next;
        cot_L = cot_n;
      } while (1);
    }
    if (BR) {
      double dr = Dot(r1, r2, r3, r4);
      cot_R = dr / cr;
      do {
        prev = Prev(Er, D);
        V(Other(prev, D), O, u1, v1);
        V(Other(prev, D), D, u2, v2);
        P1 = C2(u1, v1, u2, v2);
        if (!(P1 > eps)) break;
        cot_P = Dot(u1, v1, u2, v2) / P1;
        if (cot_P > cot_R) break;
        Remove(Er);
        Er = prev;
        cot_R = cot_P;
      } while (1);
    }
    l = Other(El, O);
    r = Other(Er, D);
    if (!BL || (BL && BR && cot_R < cot_L)) {
      B = Join(B, O, Er, r, 0);
      D = r;
    } else {
      B = Join(El, l, B, D, 0);
      O = l;
    }
  } while (1);
}

void Divide(int s, int t, edge **L, edge **R) {
  edge *a, *b, *c, *ll, *lr, *rl, *rr, *tangent;
  int n = t - s + 1;
  if (n == 2) *L = *R = Make_edge(Q[s], Q[t]);
  else if (n == 3) {
    a = Make_edge(Q[s], Q[s + 1]), b = Make_edge(Q[s + 1], Q[t]);
    Splice(a, b, Q[s + 1]);
    double v = C3(Q[s], Q[s + 1], Q[t]);
    if (v > eps) {
      c = Join(a, Q[s], b, Q[t], 0);
      *L = a;
      *R = b;
    } else if (v < -eps) {
      c = Join(a, Q[s], b, Q[t], 1);
      *L = c;
      *R = c;
    } else {
      *L = a;
      *R = b;
    }
  } else if (n > 3) {
    int split = (s + t) / 2;
    Divide(s, split, &ll, &lr);
    Divide(split + 1, t, &rl, &rr);
    Merge(lr, Q[split], rl, Q[split + 1], &tangent);
    if (Oi(tangent) == Q[s])
      ll = tangent;
    if (Dt(tangent) == Q[t])
      rr = tangent;
    *L = ll;
    *R = rr;
  }
}
void Make_Graph() {
  edge *start, *e;
  point *u, *v;
  int i;
  for (i = 0; i < n; i++) {
    u = &p[i];
    start = e = u->in;
    do {
      v = Other(e, u);
      if (u < v) {
        E[M].u = u - p, E[M].v = v - p;
        E[M++].w = dis(u, v);
        if (M >= aix * maxn) OLE();
      }
      e = Next(e, u);
    } while (e != start);
  }
}
void solve() {
  int i, test;
  scanf("%d", &test);
  while (test) {
    test--;
    n = 0;
    double ans = -1;
    scanf("%d", &n);
    for (i = 0; i < n; i++) {
      scanf("%lf%lf", &p[i].x, &p[i].y);
      p[i].index = i;
      p[i].in = NULL;
    }
    Alloc_memory();
    if (n == 1 || n == 0) {
      continue;
    }   // else RE
    sort(p, p + n);

//=========点不能有重点,有的话不满足voronoi图的性质了 
    for (i = 0; i < n; i++)
      Q[i] = p + i;
    edge *L, *R;
    Divide(0, n - 1, &L, &R);
    M = 0;
    Make_Graph();
    Kruskal();
//    puts("---------------------");
  }
}

int main() {
  freopen("input.txt", "r", stdin);
  freopen("output.txt", "w", stdout);
  solve();
  return 0;
}


\end{lstlisting}

  \subsection{四边形费马点}
    \begin{lstlisting}

//BEGIN
//POINT CLASS

typedef complex<double> Tpoint;
const double eps = 1e-8;
const double sqrt3 = sqrt(3.0);

istream& operator >>(istream& cin, Tpoint &p) {
  double x, y;
  cin >> x >> y;
  p = Tpoint(x, y);
  return cin;
}

ostream& operator <<(ostream& cout, const Tpoint &p) {
  cout << "(" << p.real() << ", " << p.imag() << ")";
  return cout;
}

int Sign(double x) {
  return fabs(x) < eps ? 0 : x > 0 ? 1 : -1;
}

bool operator ==(const Tpoint &a, const Tpoint &b) {
  return !Sign(a.real() - b.real()) && !Sign(b.imag() - a.imag());
}

bool cmp(const Tpoint &a, const Tpoint &b) {
  return  a.real() < b.real() - eps
    || (a.real() < b.real() + eps && a.imag() < b.imag());
}

double cross(const Tpoint &a, const Tpoint &b) {
  return (conj(a) * b).imag();
}

double dot(const Tpoint &a, const Tpoint &b) {
  return (conj(a) * b).real();
}

double cross(const Tpoint &a, const Tpoint &b, const Tpoint &c) {
  return cross(b - a, c - a);
}

double dot(const Tpoint &a, const Tpoint &b, const Tpoint &c) {
  return dot(b - a, c - a);
}

Tpoint unit(const Tpoint &a) {
  return a / abs(a);
}

Tpoint intersect(const Tpoint &a, const Tpoint &b, const Tpoint &c, const Tpoint &d) {
  double k1 = cross(a, b, c), k2 = cross(a, b, d);
  if (Sign(k1 - k2))
    return (c * k2 - d * k1) / (k2 - k1);
  return Tpoint(0.0, 0.0);
}

Tpoint rotate(const Tpoint &a, const Tpoint &b, const Tpoint &c) {
  Tpoint d = b - a;
  d = Tpoint(-d.imag(), d.real());
  if (Sign(cross(a, b, c)) == Sign(cross(a, b, a + d)))
    d *= -1.0;
  return unit(d);
}
//END

Tpoint p[10], a[10], b[10];
int N, T;

double totlen(const Tpoint &p, const Tpoint &a, const Tpoint &b, const Tpoint &c) {
  return abs(p - a) + abs(p - b) + abs(p - c);
}
double fermat(const Tpoint &x, const Tpoint &y, const Tpoint &z, Tpoint &cp) {
  a[0] = a[3] = x;
  a[1] = a[4] = y;
  a[2] = a[5] = z;
  double len = 1e100, len2;
  for (int i = 0; i < 3; i++) {
    len2 = totlen(a[i], x, y, z);
    if (len2 < len) {
      len = len2;
      cp = a[i];
    }
  }
  for (int i = 0; i < 3; i++) {
    b[i] = rotate(a[i + 1], a[i], a[i + 2]);
    b[i] = (a[i + 1] + a[i]) / 2.0 + b[i] * (abs(a[i + 1] - a[i]) * sqrt3 / 2.0);
  }
  b[3] = b[0];
  Tpoint cp2 = intersect(b[0], a[2], b[1], a[3]);
  len2 = totlen(cp2, x, y, z);
  if (len2 < len) {
    len = len2;
    cp = cp2;
  }
  return len;
}

double getans(const Tpoint &a) {
  double len = 0;
  for (int i = 0; i < N; i++) len += abs(a - p[i]);
  return len;
}

double mindist(const Tpoint &p, const Tpoint &a, const Tpoint &b, const Tpoint &c, const Tpoint &d) {
  return min( min(abs(p - a), abs(p - b)),
        min(abs(p - c), abs(p - d)));
}

int main() {
  N = 4;
  for (cin >> T; T; T--) {
    for (int i = 0; i < N; i++) cin >> p[i];
    Tpoint cp;
    double ret = 1e100;
    for (int i = 0; i < N; i++)
      ret = min(ret, getans(p[i]));
    for (int i = 1; i < N; i++)
      for (int j = 1; j < N; j++) if (j != i)
        for (int k = 1; k < N; k++)
          if (k != i && k != j) {
            ret = min(ret, abs(p[0] - p[i]) + abs(p[j] - p[k])
              + min(min(abs(p[0] - p[j]), abs(p[0] - p[k])),
                  min(abs(p[i] - p[j]), abs(p[i] - p[k]))));
            ret = min(ret, getans(intersect(p[0], p[i], p[j], p[k])));
          }
    for (int i = 0; i < N; i++)
      for (int j = i + 1; j < N; j++)
        for (int k = j + 1; k < N; k++) {
          double len = fermat(p[i], p[j], p[k], cp);
          ret = min(ret, len + mindist(p[6 - i - j - k], p[i], p[j], p[k], cp));
        }
    sort(p, p + N, cmp);
    Tpoint cp1, cp2;
    double len_cur, len_before;
    double len1, len2, len;
    for (int i = 1; i < N; i++) {
      cp1 = (p[0] + p[i]) / 2.0;
      int j, k;
      for (j = 1; j < N && j == i; j++);
      k = 6 - i - j;
      len_before = 1e100;
      for ( ; ; ) {
        len1 = fermat(cp1, p[j], p[k], cp2);
        len1 = fermat(cp2, p[0], p[i], cp1);
        len = len1 + abs(cp2 - p[j]) + abs(cp2 - p[k]);
        if (len < len_before - (1e-6)) len_before = len;
        else break;
      }
      ret = min(ret, len_before);
    }
    printf("%.4f\n", ret);
  }
  return 0;
}


\end{lstlisting}

  \subsection{三维计算几何基本操作}
    \begin{lstlisting}

const double EPS = 1e-7;
const double PI = 3.1415926535897932384626;
const int INF = 1000000000;

int sign(const double &a, const double &eps = EPS) {
  return a < -eps ? -1 : int(a > eps);
}

double sqr(const double &x) {
  return x * x;
}

double Sqrt(const double &x) {
  return x < 0 ? 0 : sqrt(x);
}

double arcSin(const double &a) {
  if (a <= -1.0)
    return -PI / 2;
  if (a >= 1.0)
    return PI / 2;
  return asin(a);
}

double arcCos(const double &a) {
  if (a <= -1.0)
    return PI;
  if (a >= 1.0)
    return 0;
  return acos(a);
}

struct point {
  double x, y, z;

  point() : x(0.0), y(0.0), z(0.0) {
  }

  point(double x, double y, double z) : x(x), y(y), z(z) {
  }

  point operator + (const point &rhs) const {
    return point(x + rhs.x, y + rhs.y, z + rhs.z);
  }

  point operator - (const point &rhs) const {
    return point(x - rhs.x, y - rhs.y, z - rhs.z);
  }

  point operator * (const double &k) const {
    return point(x * k, y * k, z * k);
  }

  point operator / (const double &k) const {
    return point(x / k, y / k, z / k);
  }

  double len() const {
    return sqrt(x * x + y * y + z * z);
  }

  double norm() const {
    return x * x + y * y + z * z;
  }

  point unit() const {
    double len = sqrt(x * x + y * y + z * z);
    return point(x / len, y / len, z / len);
  }

  friend double dot(const point &a, const point &b) {
    return a.x * b.x + a.y * b.y + a.z * b.z;
  }

  friend point det(const point &a, const point &b) {
    return point(a.y * b.z - a.z * b.y, a.z * b.x - a.x * b.z,
        a.x * b.y - a.y * b.x);
  }

  friend double mix(const point &a, const point &b, const point &c) {
    return a.x * b.y * c.z + a.y * b.z * c.x + a.z * b.x * c.y
        - a.z * b.y * c.x - a.x * b.z * c.y - a.y * b.x * c.z;
  }

  double distLP(const point &p1, const point &p2) const {
    return det(p2 - p1, *this - p1).len() / (p2 - p1).len();
  }

  double distFP(const point &p1, const point &p2, const point &p3) const {
    point n = det(p2 - p1, p3 - p1);
    return fabs( dot(n, *this - p1) / n.len() );
  }

};

double distLL(const point &p1, const point &p2, const point &q1, const point &q2) {
  point p = q1 - p1;
  point u = p2 - p1;
  point v = q2 - q1;
  double d = u.norm() * v.norm() - dot(u, v) * dot(u, v);
  if (sign(d) == 0)
    return p1.distLP(q1, q2);
  double s = (dot(p, u) * v.norm() - dot(p, v) * dot(u, v)) / d;
  return (p1 + u * s).distLP(q1, q2);
}

double distSS(const point &p1, const point &p2, const point &q1, const point &q2) {
  point p = q1 - p1;
  point u = p2 - p1;
  point v = q2 - q1;
  double d = u.norm() * v.norm() - dot(u, v) * dot(u, v);
  if (sign(d) == 0) {
    return min(
      min((p1 - q1).len(),
        (p1 - q2).len()
      ),
      min((p2 - q1).len(),
        (p2 - q2).len()
      ));
  }
  double s1 = (dot(p, u) * v.norm() - dot(p, v) * dot(u, v)) / d;
  double s2 = (dot(p, v) * u.norm() - dot(p, u) * dot(u, v)) / d;
  if (s1 < 0.0) s1 = 0.0;
  if (s1 > 1.0) s1 = 1.0;
  if (s2 < 0.0) s2 = 0.0;
  if (s2 > 1.0) s2 = 1.0;
  point r1 = p1 + u * s1;
  point r2 = q1 + v * s2;
  return (r1 - r2).len();
}

bool isFL(const point &p, const point &o, const point &q1, const point &q2, point &res) {
  double a = dot(o, q2 - p);
  double b = dot(o, q1 - p);
  double d = a - b;
  if (sign(d) == 0)
    return false;
  res = (q1 * a - q2 * b) / d;
  return true;
}

bool isFF(const point &p1, const point &o1, const point &p2, const point &o2, point &a, point &b) {
  point e = det(o1, o2);
  point v = det(o1, e);
  double d = dot(o2, v);
  if (sign(d) == 0)
    return false;
  point q = p1 + v * (dot(o2, p2 - p1) / d);
  a = q;
  b = q + e;
  return true;
}

\end{lstlisting}

  \subsection{凸多面体切割}
    \begin{lstlisting}

vector<vector<point> > convexCut(const vector<vector<point> > &pss, const point &p, const point &o) {
  vector<vector<point> > res;
  vector<point> sec;
  for (unsigned itr = 0, size = pss.size(); itr < size; ++itr) {
    const vector<point> &ps = pss[itr];
    int n = ps.size();
    vector<point> qs;
    bool dif = false;
    for (int i = 0; i < n; ++i) {
      int d1 = sign( dot(o, ps[i] - p) );
      int d2 = sign( dot(o, ps[(i + 1) % n] - p) );
      if (d1 <= 0) qs.push_back(ps[i]);
      if (d1 * d2 < 0) {
        point q;
        isFL(p, o, ps[i], ps[(i + 1) % n], q); // must return true
        qs.push_back(q);
        sec.push_back(q);
      }
      if (d1 == 0) sec.push_back(ps[i]);
      else dif = true;
      dif |= dot(o, det(ps[(i + 1) % n] - ps[i], ps[(i + 2) % n] - ps[i])) < -EPS;
    }
    if (!qs.empty() && dif)
      res.insert(res.end(), qs.begin(), qs.end());
  }
  if (!sec.empty()) {
    vector<point> tmp( convexHull2D(sec, o) );
    res.insert(res.end(), tmp.begin(), tmp.end());
  }
  return res;
}

vector<vector<point> > initConvex() {
  vector<vector<point> > pss(6, vector<point>(4));
  pss[0][0] = pss[1][0] = pss[2][0] = point(-INF, -INF, -INF);
  pss[0][3] = pss[1][1] = pss[5][2] = point(-INF, -INF,  INF);
  pss[0][1] = pss[2][3] = pss[4][2] = point(-INF,  INF, -INF);
  pss[0][2] = pss[5][3] = pss[4][1] = point(-INF,  INF,  INF);
  pss[1][3] = pss[2][1] = pss[3][2] = point( INF, -INF, -INF);
  pss[1][2] = pss[5][1] = pss[3][3] = point( INF, -INF,  INF);
  pss[2][2] = pss[4][3] = pss[3][1] = point( INF,  INF, -INF);
  pss[5][0] = pss[4][0] = pss[3][0] = point( INF,  INF,  INF);
  return pss;
}

\end{lstlisting}

  \subsection{三维凸包}
    \begin{lstlisting}
// please make sure there is no points coincide
namespace ConvexHull3D {

  const int MAXN = 1033;
  struct Facet {
    int a, b, c;
    Facet(int a, int b, int c): a(a), b(b), c(c) {}
  };

  #define volume(a, b, c, d) (mix(ps[b] - ps[a], ps[c] - ps[a], ps[d] - ps[a]))

  vector<Facet> getHull(int n, point ps[]) {
    static int mark[MAXN][MAXN];
    int a, b, c, stamp = 0;
    bool exist = false;
    vector<Facet> facet;

    random_shuffle(ps, ps + n);
    facet.clear();
    for (int i = 2; i < n && !exist; i++) {
      point ndir = det(ps[0] - ps[i], ps[1] - ps[i]);
      if (ndir.len() < EPS) continue;
      swap(ps[i], ps[2]);
      for (int j = i + 1; j < n; j++)
        if (sign(volume(0, 1, 2, j)) != 0) {
          swap(ps[j], ps[3]);
          facet.push_back(Facet(0, 1, 2));
          facet.push_back(Facet(0, 2, 1));
          exist = true;
          break;
        }
    }
    if (!exist) return vector<int>(); // All points are in the same plane
    for (int i = 0; i < n; ++i)
      for (int j = 0; j < n; ++j)
        mark[i][j] = 0;
    stamp = 0;

    for (int v = 3; v < n; ++v) {
      vector<Facet> tmp;
      ++stamp;
      for (unsigned i = 0; i < facet.size(); i++) {
        a = facet[i].a;
        b = facet[i].b;
        c = facet[i].c;
        if (sign(volume(v, a, b, c)) < 0)
          mark[a][b] = mark[a][c] = 
          mark[b][a] = mark[b][c] = 
          mark[c][a] = mark[c][b] = stamp;
        else
          tmp.push_back(facet[i]);
      }
      facet = tmp;
      for (unsigned i = 0; i < tmp.size(); i++) {
        a = facet[i].a;
        b = facet[i].b;
        c = facet[i].c;
        if (mark[a][b] == stamp)
          facet.push_back(Facet(b, a, v));
        if (mark[b][c] == stamp)
          facet.push_back(Facet(c, b, v));
        if (mark[c][a] == stamp)
          facet.push_back(Facet(a, c, v));
      }
    }
    return facet;
  }
  #undef volume
}

namespace Gravity {
  using ConvexHull3D::Facet;
  point findG(point ps[], const vector<Facet> &facet) {
    double ws = 0;
    point res(0.0, 0.0, 0.0);
    point o = ps[ facet[0].a ];
    for (int i = 0, size = facet.size(); i < size; ++i) {
      const point &a = ps[ facet[i].a ];
      const point &b = ps[ facet[i].b ];
      const point &c = ps[ facet[i].c ];
      point p = (a + b + c + o) * 0.25;
      double w = mix(a - o, b - o, c - o);
      ws += w;
      res = res + p * w;
    }
    res = res / ws;
    return res;
  }
}

\end{lstlisting}

  \subsection{球面点表面点距离}
    \begin{lstlisting}
double distOnBall(double lati1, double longi1, double lati2, double longi2, double R) {
  lati1 *= PI / 180;
  longi1 *= PI / 180;
  lati2 *= PI / 180;
  longi2 *= PI / 180;
  double x1 = cos(lati1) * sin(longi1);
  double y1 = cos(lati1) * cos(longi1);
  double z1 = sin(lati1);
  double x2 = cos(lati2) * sin(longi2);
  double y2 = cos(lati2) * cos(longi2);
  double z2 = sin(lati2);

  double theta = acos(x1 * x2 + y1 * y2 + z1 * z2);
  return R * theta;
}

\end{lstlisting}
    
  \subsection{长方体表面点距离}
    \begin{lstlisting}
namespace DistOnCuboid {
  int r;
  void turn(int i, int j, int x, int y, int z, int x0, int y0, int L, int W, int H) {
    if (z == 0) {
      int R = x * x + y * y;
      if (R < r) r = R;
    }
    else {
      if (i >= 0 && i < 2)
        turn(i + 1, j, x0 + L + z, y, x0 + L - x, x0 + L, y0, H, W, L);
      if (j >= 0 && j < 2)
        turn(i, j + 1, x, y0 + W + z, y0 + W - y, x0, y0 + W, L, H, W);
      if (i <= 0 && i > -2)
        turn(i - 1, j, x0 - z, y, x - x0, x0 - H, y0, H, W, L);
      if (j <= 0 && j > -2)
        turn(i, j - 1, x, y0 - z, y - y0, x0, y0 - H, L, H, W);
    }
  }
  int calc(int L, int H, int W, int x1, int y1, int z1, int x2, int y2, int z2) {
    if (z1 != 0 && z1 != H)
      if (y1 == 0 || y1 == W)
        swap(y1, z1), swap(y2, z2), swap(W, H);
      else
        swap(x1, z1), swap(x2, z2), swap(L, H);
    if (z1 == H)
      z1 = 0, z2 = H - z2;
    r = 0x3fffffff;
    turn(0, 0, x2 - x1, y2 - y1, z2, -x1, -y1, L, W, H);
    return r;
  }
}

\end{lstlisting}

  \subsection{最小覆盖球}
    \begin{lstlisting}

namespace MinBall {

int outCnt;
point out[4], res;
double radius;

void ball() {
  static point q[3];
  static double m[3][3], sol[3], L[3], det;
  int i, j;

  res = point(0.0, 0.0, 0.0);
  radius = 0.0;

  switch (outCnt) {
  case 1:
    res = out[0];
    break;
  case 2:
    res = (out[0] + out[1]) * 0.5;
    radius = (res - out[0]).norm();
    break;
  case 3:
    q[0] = out[1] - out[0];
    q[1] = out[2] - out[0];
    for (i = 0; i < 2; ++i)
      for (j = 0; j < 2; ++j)
        m[i][j] = dot(q[i], q[j]) * 2.0;
    for (i = 0; i < 2; ++i)
      sol[i] = dot(q[i], q[i]);
    det = m[0][0] * m[1][1] - m[0][1] * m[1][0];
    if (sign(det) == 0)
      return;
    L[0] = (sol[0] * m[1][1] - sol[1] * m[0][1]) / det;
    L[1] = (sol[1] * m[0][0] - sol[0] * m[1][0]) / det;
    res = out[0] + q[0] * L[0] + q[1] * L[1];
    radius = (res - out[0]).norm();
    break;
  case 4:
    q[0] = out[1] - out[0];
    q[1] = out[2] - out[0];
    q[2] = out[3] - out[0];
    for (i = 0; i < 3; ++i)
      for (j = 0; j < 3; ++j)
        m[i][j] = dot(q[i], q[j]) * 2;
    for (i = 0; i < 3; ++i)
      sol[i] = dot(q[i], q[i]);
    det = m[0][0] * m[1][1] * m[2][2] + m[0][1] * m[1][2] * m[2][0]
        + m[0][2] * m[2][1] * m[1][0] - m[0][2] * m[1][1] * m[2][0]
        - m[0][1] * m[1][0] * m[2][2] - m[0][0] * m[1][2] * m[2][1];
    if (sign(det) == 0)
      return;
    for (j = 0; j < 3; ++j) {
      for (i = 0; i < 3; ++i)
        m[i][j] = sol[i];
      L[j] = (m[0][0] * m[1][1] * m[2][2] + m[0][1] * m[1][2] * m[2][0]
          + m[0][2] * m[2][1] * m[1][0] - m[0][2] * m[1][1] * m[2][0]
          - m[0][1] * m[1][0] * m[2][2] - m[0][0] * m[1][2] * m[2][1])
          / det;
      for (i = 0; i < 3; ++i)
        m[i][j] = dot(q[i], q[j]) * 2;
    }
    res = out[0];
    for (i = 0; i < 3; ++i)
      res += q[i] * L[i];
    radius = (res - out[0]).norm();
  }
}

void minball(int n, point pt[]) {
  ball();
  if (outCnt < 4)
    for (int i = 0; i < n; ++i)
      if ((res - pt[i]).norm() > +radius + EPS) {
        out[outCnt] = pt[i];
        ++outCnt;
        minball(i, pt);
        --outCnt;
        if (i > 0) {
          point Tt = pt[i];
          memmove(&pt[1], &pt[0], sizeof(point) * i);
          pt[0] = Tt;
        }
      }
}

pair<point, double> main(int npoint, point pt[]) { // 0-based
  random_shuffle(pt, pt + npoint);
  radius = -1;
  for (int i = 0; i < npoint; i++) {
    if ((res - pt[i]).norm() > EPS + radius) {
      outCnt = 1;
      out[0] = pt[i];
      minball(i, pt);
    }
  }
  return make_pair(res, sqrt(radius));
}

}

\end{lstlisting}

\section{数据结构}
  \subsection{动态凸包(只支持插入)}
    \begin{lstlisting}

#include <cstdio>
#include <cstdlib>
#include <cassert>
#include <map>

#define x first
#define y second

using namespace std;

typedef long long LL;
typedef map<int, int> mii;
typedef map<int, int>::iterator mit;

inline int sign(LL x) {
  return x < 0 ? -1 : int(x > 0);
}

struct point {
  int x, y;

  point(): x(0), y(0) {
  }

  point(const int &x, const int &y) : x(x), y(y) {
  }

  point(const mit &p): x(p->first), y(p->second) {
  }

  friend inline point operator - (const point &a, const point &b) {
    return point(a.x - b.x, a.y - b.y);
  }

  friend inline LL det(const point &a, const point &b) {
    return (LL)a.x * b.y - (LL)a.y * b.x;
  }
};

inline bool checkInside(mii &a, const point &p) { // border inclusive
  int x = p.x, y = p.y;
  mit p1 = a.lower_bound(x);
  if (p1 == a.end()) return false;
  if (p1->x == x)  return y <= p1->y;
  if (p1 == a.begin()) return false;
  mit p2(p1--);
  return sign(det(p - point(p1), point(p2) - p)) >= 0;
}

inline void addPoint(mii &a, const point &p) { // no collinear points
  int x = p.x, y = p.y;
//  assert(!checkInside(a, p));

  mit pnt = a.insert(make_pair(x, y)).first, p1, p2;
  pnt->y = y;

  for ( ; ; a.erase(p2)){
    p1 = pnt;
    if (++p1 == a.end()) break;
    p2 = p1;
    if (++p1 == a.end()) break;
    if (det(point(p2) - p, point(p1) - p) < 0) break;
  }
  for ( ; ; a.erase(p2)){
    p1 = pnt;
    if (p1 == a.begin()) break;
    --p1;
    if (p1 == a.begin()) break;
    p2 = p1--;
    if (det(point(p2) - p, point(p1) - p) > 0) break;
  }
}

int main() {
  int N, t, x, y;
  mii upperHull, lowerHull;

  scanf("%d", &N);
  for (int i = 1; i <= N; ++i) {
    scanf("%d %d %d", &t, &x, &y);
    if (t == 1) {
      if (!checkInside(upperHull, point(x,  y))) addPoint(upperHull, point(x,  y));
      if (!checkInside(lowerHull, point(x, -y))) addPoint(lowerHull, point(x, -y));
    }
    else {
      if (checkInside(upperHull, point(x, y)) && checkInside(lowerHull, point(x, -y)))
        puts("YES");
      else
        puts("NO");
    }
  }
  return 0;
}

\end{lstlisting}
    
  \subsection{Rope用法}
    \begin{lstlisting}
#include <ext/rope>

using __gnu_cxx::crope;
using __gnu_cxx::rope;

a = b.substr(from, len);        // [from, from + len)
a = b.substr(from);             // [from, from]
a = b.c_str();                  // quite slow and might lead to memory leaks
a.insert(p, str);               // insert str before position p
a.erase(i, n);                  // erase [i, i + n)

\end{lstlisting}
  
  \subsection{可持久化Treap}
    \begin{lstlisting}

inline bool randomBySize(int a, int b) {
  return rand() % (a + b) < a;
}

tree merge(tree x, tree y) {
  if (x == null) return y;
  if (y == null) return x;
  tree t = NULL;
  if (randomBySize(x->size, y->size)) {
    t = newNode(x);
    t->r = merge(x->r, y);
  }
  else {
    t = newNode(y);
    t->l = merge(x, y->l);
  }
  update(t);
  return t;
}

void splitByKey(tree t, int k, tree &l, tree &r) { // [-oo, k) [k, +oo)
  if (t == null) l = r = null;
  else if (t->key < k) {
    l = newNode(t);
    splitByKey(t->r, k, l->r, r);
    update(l);
  }
  else {
    r = newNode(t);
    splitByKey(t->l, k, l, r->l);
    update(r);
  }
}

\end{lstlisting}
  
  \subsection{Link-Cut Tree}
    \begin{lstlisting}

struct node {
  int rev;
  node *pre, *ch[2];
} base[MAXN], nil, *null;
typedef node *tree;

#define isRoot(x) (x->pre->ch[0] != x && x->pre->ch[1] != x)
#define isRight(x) (x->pre->ch[1] == x)

inline void MakeRev(tree t) {
  if (t != null) {
    t->rev ^= 1;
    swap(t->ch[0], t->ch[1]);
  }
}

inline void Update(tree t) {
}

inline void PushDown(tree t) {
  if (t->rev) {
    MakeRev(t->ch[0]);
    MakeRev(t->ch[1]);
    t->rev = 0;
  }
}

inline void Rotate(tree x) {
  tree y = x->pre;
  PushDown(y); PushDown(x);
  int d = isRight(x);
  if (!isRoot(y)) y->pre->ch[isRight(y)] = x;
  x->pre = y->pre;
  if ((y->ch[d] = x->ch[!d]) != null) y->ch[d]->pre = y;
  x->ch[!d] = y;
  y->pre = x;
  Update(y);
}

inline void Splay(tree x) {
  PushDown(x);
  for (tree y; !isRoot(x); Rotate(x)) {
    y = x->pre;
    if (!isRoot(y))
      Rotate(isRight(x) != isRight(y) ? x : y);
  }
  Update(x);
}
inline void Splay(tree x, tree to) {
  PushDown(x);
  for (tree y; (y = x->pre) != to; Rotate(x))
    if (y->pre != to)
      Rotate(isRight(x) != isRight(y) ? x : y);
  Update(x);
}

inline tree Access(tree t) {
  tree last = null;
  while (t != null) {
    Splay(t);
    t->ch[1] = last;
    Update(t);
    last = t;
    t = t->pre;
  }
  return last;
}

inline void MakeRoot(tree t) {
  Access(t);
  Splay(t);
  MakeRev(t);
}

inline tree FindRoot(tree t) {
  Access(t);
  Splay(t);
  tree last = null;
  for (; t != null;) {
    PushDown(t); // important
    last = t;
    t = t->ch[0];
  }
  Splay(last);
  return last;
}

inline void Cut(tree t) {
  Access(t);
  Splay(t);
  t->ch[0]->pre = null;
  t->ch[0] = null;
  Update(t);
}
inline void Cut_simp(tree a, tree b) //simple version
{
  MakeRoot(a);
  Access(a);
  Splay(b);
  b -> pre = null;
}
inline void Cut(tree x, tree y) {
  tree upper = (Access(x), Access(y));
  if (upper == x) {
    Splay(x);
    y->pre = null;
    x->ch[1] = null;
    Update(x);
  } else if (upper == y) {
    Access(x);
    Splay(y);
    x->pre = null;
    y->ch[1] = null;
    Update(y);
  } else
    assert(0); // impossible to happen
}

inline void Join(tree x, tree y) {
  MakeRoot(y);
  y->pre = x;
}

// query the cost in path a <-> b, lca inclusive
inline int Query(tree a, tree b) { 
  Access(a);
  tree c = Access(b); // c is lca
  int v1 = c->ch[1]->maxCost;
  Access(a);
  int v2 = c->ch[1]->maxCost;
  return max(max(v1, v2), c->cost);
}

void Init() {
  null = &nil;
  null->ch[0] = null->ch[1] = null->pre = null;
  null->rev = 0;
  Rep(i, 1, N) {
    node &n = base[i];
    n.rev = 0;
    n.pre = n.ch[0] = n.ch[1] = null;
  }
}


\end{lstlisting}

  \subsection{K-D Tree Nearest}
    \begin{lstlisting}
struct Rectangle;

struct point {
  int x, y;

  point() : x(0), y(0) {
  }

  point(int x, int y) : x(x), y(y) {
  }

  bool on(const point &, const point &) const;

  bool in(point *, int) const;

  int operator [] (int i) const {
    return i ? y : x;
  }

  int& operator [] (int i) {
    return i ? y : x;
  }

  LL norm() const {
    return (LL)x * x + (LL)y * y;
  }

  LL to(const Rectangle&) const;

  point operator - (const point &b) const {
    return point(x - b.x, y - b.y);
  }
};

inline LL sqr(LL x) {
  return x * x;
}

inline LL dot(const point &a, const point &b) {
  return (LL)a.x * b.x + (LL)a.y * b.y;
}

inline LL det(const point &a, const point &b) {
  return (LL)a.x * b.y - (LL)a.y * b.x;
}

inline bool point::on(const point &a, const point &b) const {
  const point &p = *this;
  return det(a - p, b - p) == 0 && dot(a - p, b - p) <= 0;
}

inline bool point::in(point *polygon, int n) const {
  const point &p = *this;
  int cnt = 0;
  for (int i = 0; i < n; ++i) {
    const point &a = polygon[i], &b = polygon[(i + 1) % n];
    if (p.on(a, b)) return true;
    int d0 = sign(det(b - a, p - a));
    int d1 = a.y - p.y;
    int d2 = b.y - p.y;
    cnt += d0 > 0 && d1 <= 0 && d2 > 0;
    cnt -= d0 < 0 && d2 <= 0 && d1 > 0;
  }
  return cnt != 0;
}

struct Rectangle {
  int min[2], max[2];
  Rectangle() {
    min[0] = min[1] = INT_MAX;
    max[0] = max[1] = INT_MIN;
  }

  void add(const point &p) {
    for (int i = 0; i < 2; ++i) {
      min[i] = std::min(min[i], p[i]);
      max[i] = std::max(max[i], p[i]);
    }
  }
};

LL point::to(const Rectangle &r) const {
  const point &p = *this;
  LL res = 0;
  for (int i = 0; i < 2; ++i)
    res += sqr( min(max(p[i], r.min[i]), r.max[i]) - p[i] );
  return res;
}

const int MAXN = 20033;
int n, pivot, seperator[MAXN * 2 + 1];
vector<int> order;
point points[MAXN], polygon[33];
Rectangle rec[MAXN * 2 + 1];

inline int getId(int l, int r) {
  return (l + r) | (l != r);
}

bool compare(int i, int j) {
  if (points[i][pivot] != points[j][pivot])
    return points[i][pivot] < points[j][pivot];
  return i < j;
}

void build(int l, int r, int type) {
  int id = getId(l, r);
  rec[id] = Rectangle();
  Rep(i, l, r) rec[id].add( points[ order[i] ] );
  if (l < r) {
    int m = (l + r) >> 1;
    pivot = type;
    nth_element(order.begin() + l, order.begin() + m, order.begin() + r + 1, compare);
    seperator[id] = order[m];
    build(l, m, type ^ 1);
    build(m + 1, r, type ^ 1);
  }
}

priority_queue<pair<LL, int> > answer;

void query(int l, int r, int type) {
  const point &p = points[n];
  int id = getId(l, r);
  if (answer.size() == 2 && p.to(rec[id]) > answer.top().first)
    return;
  if (l == r) {
    answer.push( make_pair((p - points[order[l]]).norm(), order[l] ) );
    if (answer.size() > 2) answer.pop();
  }
  else {
    int m = (l + r) >> 1;
    pivot = type;
    int dir = compare(seperator[id], n);
    if (dir)
      query(l, m, type ^ 1);
    query(m + 1, r, type ^ 1);
    if (!dir)
      query(l, m, type ^ 1);
  }
}

void queryTree(int size) {
  answer = priority_queue<pair<LL, int> >();
  scanf("%d %d", &points[n].x, &points[n].y);
  query(0, size - 1, 0);
  vector<pair<LL, int> > buffer;
  while (!answer.empty()) {
    buffer.push_back( answer.top() );
    answer.pop();
  }
  sort(buffer.begin(), buffer.end());
  printf("%d %d\n", buffer[0].second + 1, buffer[1].second + 1);
}

void solve() {
  int R;

  scanf("%d", &n);
  Foru(i, 0, n) scanf("%d %d", &points[i].x, &points[i].y);
  scanf("%d", &R);

  Rep(region, 1, R) {
    printf("Region %d\n", region);

    int B;
    scanf("%d", &B);
    Foru(i, 0, B) scanf("%d %d", &polygon[i].x, &polygon[i].y);

    order.clear();
    Foru(i, 0, n)
      if (points[i].in(polygon, B))
        order.push_back(i);
    int size = order.size();
    build(0, size - 1, 0);
    order.push_back(n);
    int M;
    scanf("%d", &M);
    while (M--)
      queryTree(size);
  }
}

int main() {
  int T;
  scanf("%d", &T);
  Rep(Case, 1, T) {
    printf("Case #%d:\n", Case);
    solve();
  }
  return 0;
}


\end{lstlisting}

  \subsection{K-D Tree Farthest}
    \input{"Data Structure/KDTree_2.tex"}

  \subsection{树链剖分}
    \begin{lstlisting}

struct enode {
  int to;
  enode *next;
} ebase[MAXN * 2], *etop, *fir[MAXN];
typedef enode *edge;

inline void addEdge(int a, int b) {
  etop->to = b;
  etop->next = fir[a];
  fir[a] = etop++;
}

int N;

namespace TreeDecomposition {

  int fa[MAXN], dep[MAXN], Q[MAXN], size[MAXN], own[MAXN];

  void Decomposion() {
    static int path[MAXN];
    int x, y, a, next, head = 0, tail = 0, cnt;
    for (fa[1] = -1, dep[1] = 0, Q[++tail] = 1; head < tail;) {
      x = Q[++head];
      for (edge e(fir[x]); e; e = e->next)
        if ( (y = e->to) != fa[x] ) {
          dep[y] = dep[x] + 1;
          fa[y] = x;
          Q[++tail] = y;
        }
    }
    for (int i = N; i >= 1; --i) {
      int &res = size[x = Q[i]] = 1;
      for (edge e(fir[x]); e; e = e->next) if ( (y = e->to) != fa[x] )
        res += size[y];
    }
    for (int i = 1; i <= N; ++i) own[i] = -1;
    for (int i = 1; i <= N; ++i) if (own[a = Q[i]] == -1)
      for (x = a, cnt = 0; ; x = next) {
        next = -1;
        own[x] = a;
        path[++cnt] = x; // heavy path
        for (edge e(fir[x]); e; e = e->next) if ( (y = e->to) != fa[x] )
          if (next == -1 || size[y] > size[next])
            next = y;
        if (next == -1) {
          tree[a].init(cnt, path); // init the segment tree
          break;
        }
      }
  }
}

\end{lstlisting}

\section{字符串相关}
  \subsection{Manacher}
    \begin{lstlisting}

// len[i] : the max length of palindrome whose mid point is (i / 2)
void Manacher(int n, char cs[], int len[]) { // 0-based, len[] must be double sized
  for (int i = 0; i < n + n; ++i) len[i] = 0;
  for (int i = 0, j = 0, k; i < n * 2; i += k, j = max(j - k, 0)) {
    while (i - j >= 0 && i + j + 1 < n * 2 && cs[(i - j) / 2] == cs[(i + j + 1) / 2]) j++;
    len[i] = j;
    for (k = 1; i - k >= 0 && j - k >= 0 && len[i - k] != j - k; k++) {
      len[i + k] = min(len[i - k], j - k);
    }
  }
}

\end{lstlisting}

    \subsection{Aho-Corasick 自动机}
    \begin{lstlisting}

void construct() {
  static tree Q[MAX_NODE];
  int head = 0, tail = 0;
  root->fail = root;
  for (Q[++tail] = root; head < tail; ) {
    tree x = Q[++head];
//    if (x->fail->danger) x->danger = true;
    Rep(d, 0, sigma - 1) if (!x->next[d])
      x->next[d] = (x == root) ? (root) : (x->fail->next[d]);
    else {
      x->next[d]->fail = (x == root) ? (root) : (x->fail->next[d]);
      Q[++tail] = x->next[d];
    }
  }
}
\end{lstlisting}

  \subsection{后缀自动机}
    \begin{lstlisting}

const int MAXLEN = 100033;
const int MAXNODE = MAXLEN * 2 + 1;

struct node {
  int len, opt, ans;
  node *fa, *go[26];
} base[MAXNODE], *top = base, *root;
typedef node *tree;

int N, Ncnt;
char A[MAXLEN];
tree Q[MAXNODE];

inline tree newNode(int len) {
  top->len = top->ans = len;
  top->fa = NULL;
  top->opt = 0;
  memset(top->go, 0, sizeof(top->go));
  return top++;
}

inline tree newNode(int len, tree fa, tree *go) {
  top->len = top->ans = len;
  top->fa = fa;
  top->opt = 0;
  memcpy(top->go, go, sizeof(top->go));
  return top++;
}

void construct() {
  tree p, q, up, fa;
  N = strlen(A);
  p = root = newNode(0);
  for (int i = 0; i < N; ++i) {
    int w = A[i] - 'a';
    up = p; p = newNode(i + 1);
    for ( ; up && !up->go[w]; up = up->fa)
      up->go[w] = p;
    if (!up) p->fa = root;
    else {
      q = up->go[w];
      if (up->len + 1 == q->len) p->fa = q;
      else {
        fa = newNode(up->len + 1, q->fa, q->go);
        p->fa = q->fa = fa;
        for ( ; up && up->go[w] == q; up = up->fa)
          up->go[w] = fa;
      }
    }
  }
  static int cnt[MAXLEN];
  memset(cnt, 0, sizeof(cnt));
  Ncnt = top - base;
  for (tree i(base); i != top; ++i) ++cnt[i->len];
  Rep(i, 1, N) cnt[i] += cnt[i - 1];
  for (tree i(base); i != top; ++i) Q[ cnt[i->len]-- ] = i;
}

void work() {
  scanf("%s", A);
  construct();
  while (scanf("%s", A) != EOF) {
    int len = 0;
    tree t = root;
    for (int i(0), Len(strlen(A)); i < Len; ++i) {
      int w = A[i] - 'a';
      if (t->go[w]) {
        ++len;
        t = t->go[w];
      } else {
        for ( ; t && !t->go[w]; t = t->fa);
        if (t) {
          len = t->len + 1;
          t = t->go[w];
        } else {
          len = 0;
          t = root;
        }
      }
      Up(t->opt, len);
    }
    Til(i, Ncnt, 1) {
      tree x = Q[i], y = x->fa;
      if (y) Up(y->opt, x->opt);
      Down(x->ans, x->opt);
      x->opt = 0;
    }
  }
  int ans = 0;
  for (tree i(base); i != top; ++i) Up(ans, i->ans);
  printf("%d\n", ans);
}

int main() {
  solve();
  return 0;
}

\end{lstlisting}

  \subsection{后缀数组}
    \begin{lstlisting}
// 待排序的字符串放在 r[0 ... n - 1] 中,最大值小于 m
// r[0 ... n - 2] > 0, r[n - 1] = 0
// 结果放在  sa[0 ... n - 1]

namespace SuffixArrayDoubling {
  int wa[MAXN], wb[MAXN], wv[MAXN], ws[MAXN];

  int cmp(int *r, int a, int b, int l) {
    return r[a] == r[b] && r[a + l] == r[b + l];
  }

  void da(int *r, int *sa, int n, int m) {
    int i, j, p, *x = wa, *y = wb, *t;
    for (i = 0; i < m; i++)
      ws[i] = 0;
    for (i = 0; i < n; i++)
      ws[x[i] = r[i]]++;
    for (i = 1; i < m; i++)
      ws[i] += ws[i - 1];
    for (i = n - 1; i >= 0; i--)
      sa[--ws[x[i]]] = i;
    for (j = 1, p = 1; p < n; j *= 2, m = p) {
      for (p = 0, i = n - j; i < n; i++)
        y[p++] = i;
      for (i = 0; i < n; i++)
        if (sa[i] >= j)
          y[p++] = sa[i] - j;
      for (i = 0; i < n; i++)
        wv[i] = x[y[i]];
      for (i = 0; i < m; i++)
        ws[i] = 0;
      for (i = 0; i < n; i++)
        ws[wv[i]]++;
      for (i = 1; i < m; i++)
        ws[i] += ws[i - 1];
      for (i = n - 1; i >= 0; i--)
        sa[--ws[wv[i]]] = y[i];
      for (t = x, x = y, y = t, p = 1, x[sa[0]] = 0, i = 1; i < n; i++)
        x[sa[i]] = cmp(y, sa[i - 1], sa[i], j) ? p - 1 : p++;
    }
  }
}

// 各个参数的作用同倍增算法数组的
// r[] 与  sa[] 大小需 3 * n
namespace SuffixArrayDC3 {

  #define F(x) ((x) / 3 + ((x) % 3 == 1 ? 0 : tb))
  #define G(x) ((x) < tb ? (x) * 3 + 1 : ((x) - tb) * 3 + 2)

  int wa[MAXN], wb[MAXN], wv[MAXN], ws[MAXN];
  int c0(int *r, int a, int b) {
    return r[a] == r[b] && r[a + 1] == r[b + 1] && r[a + 2] == r[b + 2];
  }

  int c12(int k, int *r, int a, int b) {
    if (k == 2)
      return r[a] < r[b] || (r[a] == r[b] && c12(1, r, a + 1, b + 1));
    else
      return r[a] < r[b] || (r[a] == r[b] && wv[a + 1] < wv[b + 1]);
  }

  void sort(int *r, int *a, int *b, int n, int m) {
    int i;
    for (i = 0; i < n; i++)
      wv[i] = r[a[i]];
    for (i = 0; i < m; i++)
      ws[i] = 0;
    for (i = 0; i < n; i++)
      ws[wv[i]]++;
    for (i = 1; i < m; i++)
      ws[i] += ws[i - 1];
    for (i = n - 1; i >= 0; i--)
      b[--ws[wv[i]]] = a[i];
    return;
  }

  void dc3(int *r, int *sa, int n, int m) {
    int i, j, *rn = r + n, *san = sa + n, ta = 0, tb = (n + 1) / 3, tbc = 0, p;
    r[n] = r[n + 1] = 0;
    for (i = 0; i < n; i++)
      if (i % 3 != 0)
        wa[tbc++] = i;
    sort(r + 2, wa, wb, tbc, m);
    sort(r + 1, wb, wa, tbc, m);
    sort(r, wa, wb, tbc, m);
    for (p = 1, rn[F(wb[0])] = 0, i = 1; i < tbc; i++)
      rn[F(wb[i])] = c0(r, wb[i - 1], wb[i]) ? p - 1 : p++;
    if (p < tbc)
      dc3(rn, san, tbc, p);
    else
      for (i = 0; i < tbc; i++)
        san[rn[i]] = i;
    for (i = 0; i < tbc; i++)
      if (san[i] < tb)
        wb[ta++] = san[i] * 3;
    if (n % 3 == 1)
      wb[ta++] = n - 1;
    sort(r, wb, wa, ta, m);
    for (i = 0; i < tbc; i++)
      wv[wb[i] = G(san[i])] = i;
    for (i = 0, j = 0, p = 0; i < ta && j < tbc; p++)
      sa[p] = c12(wb[j] % 3, r, wa[i], wb[j]) ? wa[i++] : wb[j++];
    for (; i < ta; p++)
      sa[p] = wa[i++];
    for (; j < tbc; p++)
      sa[p] = wb[j++];
  }

  #undef F
  #undef G
}

namespace CalcHeight {
  int rank[MAXN], height[MAXN];
  void calheight(int *r, int *sa, int n) {
    int i, j, k = 0;
    for (i = 1; i <= n; i++)
      rank[sa[i]] = i;
    for (i = 0; i < n; height[rank[i++]] = k)
      for (k ? k-- : 0, j = sa[rank[i] - 1]; r[i + k] == r[j + k]; k++);
    return;
  }
}

\end{lstlisting}
  
  \subsection{环串最小表示}
    \input{"String Related/Minimial Representation Of A Cyclic String.tex"}

\section{图论}
  \subsection{带花树}
    \begin{lstlisting}

namespace Blossom {

  int n, match[MAXN];
  int Q[MAXN], head, tail;
  int pred[MAXN], label[MAXN];
  int inq[MAXN], inb[MAXN];
  int S, T, lca;
  vector<int> link[MAXN];

  void init(int _n) {
    n = _n;
    for (int i = 0; i < n; ++i)
      link[i].clear();
  }

  inline void push(int x) {
    Q[tail++] = x;
    inq[x] = true;
  }

  int findCommonAncestor(int x, int y) {
    static bool inPath[MAXN];
    for (int i = 0; i < n; ++i)
      inPath[i] = 0;
    for ( ; ; x = pred[ match[x] ]) {
      x = label[x];
      inPath[x] = true;
      if (x == S) break;
    }
    for ( ; ; y = pred[ match[y] ]) {
      y = label[y];
      if (inPath[y]) break;
    }
    return y;
  }

  void resetTrace(int x, int lca) {
    while (label[x] != lca) {
      int y = match[x];
      inb[ label[x] ] = inb[ label[y] ] = true;
      x = pred[y];
      if (label[x] != lca) pred[x] = y;
    }
  }

  void blossomContract(int x, int y) {
    lca = findCommonAncestor(x, y);
    for (int i = 0; i < n; ++i)
      inb[i] = 0;
    resetTrace(x, lca);
    resetTrace(y, lca);
    if (label[x] != lca) pred[x] = y;
    if (label[y] != lca) pred[y] = x;
    for (int i = 0; i < n; ++i)
      if (inb[ label[i] ]) {
        label[i] = lca;
        if (!inq[i]) push(i);
      }
  }

  bool findAugmentingPath() {
    for (int i = 0; i < n; ++i) pred[i] = -1;
    for (int i = 0; i < n; ++i) label[i] = i;
    for (int i = 0; i < n; ++i) inq[i] = 0;

    int x, y, z;
    head = tail = 0;
    for (push(S); head < tail; ) {
      x = Q[head++];
      for (int i = (int)link[x].size() - 1; i >= 0; --i) {
        y = link[x][i];
        if (label[x] == label[y] || x == match[y]) continue;
        if (y == S || (match[y] >= 0 && pred[ match[y] ] >= 0))
          blossomContract(x, y);
        else if (pred[y] == -1) {
          pred[y] = x;
          if (match[y] >= 0)
            push(match[y]);
          else { // augment
            for (x = y; x >= 0; x = z) {
              y = pred[x];
              z = match[y];
              match[x] = y;
              match[y] = x;
            }
            return true;
          }
        }
      }
    }
    return false;
  }

  int findMaxMatching() {
    int ans = 0;
    for (int i = 0; i < n; ++i) match[i] = -1;
    for (S = 0; S < n; ++S)
      if (match[S] == -1)
        if (findAugmentingPath())
          ++ans;
    return ans;
  }

}

\end{lstlisting}

  \subsection{最大流}
    \begin{lstlisting}

namespace Maxflow {

  int h[MAXNODE], vh[MAXNODE];
  int S, T, Ncnt;
  edge cur[MAXNODE], pe[MAXNODE];

  void init(int _S, int _T, int _Ncnt) {
    S = _S;
    T = _T;
    Ncnt = _Ncnt;
  }

  int maxflow() {
    static int Q[MAXNODE];
    int x, y, augc, flow = 0, head = 0, tail = 0;
    edge e;
    
    Rep(i, 0, Ncnt) cur[i] = fir[i];
    Rep(i, 0, Ncnt) h[i] = INF;
    Rep(i, 0, Ncnt) vh[i] = 0;

    for (Q[++tail] = T, h[T] = 0; head < tail; ) {
      x = Q[++head]; ++vh[ h[x] ];
      for (e = fir[x]; e; e = e->next) if (e->op->c)
        if (h[y = e->to] >= INF) {
          h[y] = h[x] + 1;
          Q[++tail] = y;
        }
    }

    for (x = S; h[S] < Ncnt; ) {
      for (e = cur[x]; e; e = e->next) if (e->c)
        if (h[y = e->to] + 1 == h[x]) {
          cur[x] = pe[y] = e;
          x = y;
          break;
        }
      if (!e) {
        if (--vh[ h[x] ] == 0) break;
        h[x] = Ncnt; cur[x] = NULL;
        for (e = fir[x]; e; e = e->next) if (e->c)
          if ( cMin( h[x], h[e->to] + 1  ) )
            cur[x] = e;
        ++vh[ h[x] ];
        if (x != S) x = pe[x]->op->to;
      }
      else if (x == T) {
        augc = INF;
        for (x = T; x != S; x = pe[x]->op->to) cMin(augc, pe[x]->c);
        for (x = T; x != S; x = pe[x]->op->to) {
          pe[x]->c -= augc;
          pe[x]->op->c += augc;
        }
        flow += augc;
      }
    }
    return flow;
  }
}


\end{lstlisting}
  
  \subsection{KM}
    \begin{lstlisting}

namespace KM {

  int n, m; // left && right
  int ans;
  int L[MAX_LEFT], R[MAX_RIGHT], v[MAX_RIGHT];
  bool bx[MAX_LEFT], by[MAX_RIGHT];

  bool find(int x) {
    bx[x] = true;
    for (edge e(fir[x]); e; e = e->next) {
      int y = e->to, c = e->c;
      if (!by[y] && L[x] + R[y] == c) {
        by[y] = true;
        if (!v[y] || find(v[y])) {
          v[y] = x;
          return true;
        }
      }
    }
    return false;
  }

  int km() {
    memset(L, 0, sizeof(L));
    memset(R, 0, sizeof(R));
    memset(v, 0, sizeof(v));
    for (int x = 1; x <= n; ++x)
      for (edge e(fir[x]); e; e = e->next)
        L[x] = max(L[x], e->c);
    ans = 0;
    for (int i = 1; i <= min(n, m); ++i)
      for ( ; ; ) {
        memset(bx, 0, sizeof(bx));
        memset(by, 0, sizeof(by));
        if (find(i)) break;
        int Min = INF;
        for (int x = 1; x <= n; ++x) if (bx[x])
          for (edge e(fir[x]); e; e = e->next) {
            int y = e->to;
            if (!by[y])
              Min = min(Min, L[x] + R[y] - e->c);
          }
        for (int x = 1; x <= n; ++x) if (bx[x]) L[x] -= Min;
        for (int y = 1; y <= m; ++y) if (by[y]) R[y] += Min;
      }
    for (int x = 1; x <= n; ++x)
      for (edge e(fir[x]); e; e = e->next)
        if (v[e->to] == x)
          ans += e->c;
    return ans;
  }
}

\end{lstlisting}

  \subsection{2-SAT 与 Kosaraju}
    \begin{lstlisting}

// reverse edge is needed
namespace SCC {
  int code[MAXN * 2], seq[MAXN * 2], sCnt;
  void DFS_1(int x) {
    code[x] = 1;
    for (edge e(fir[x]); e; e = e->next)
      if (code[e->to] == -1)
        DFS_1(e->to);
    seq[++sCnt] = x;
  }

  void DFS_2(int x) {
    code[x] = sCnt;
    for (edge e(fir2[x]); e; e = e->next)
      if (code[e->to] == -1)
        DFS_2(e->to);
  }

  void SCC(int N) {
    sCnt = 0;
    for (int i = 1; i <= N; ++i) code[i] = -1;
    for (int i = 1; i <= N; ++i)
      if (code[i] == -1)
        DFS_1(i);

    sCnt = 0;
    for (int i = 1; i <= N; ++i) code[i] = -1;
    for (int i = N; i >= 1; --i)
      if (code[seq[i]] == -1) {
        ++sCnt;
        DFS_2(seq[i]);
      }
  }
}

// true    -   2i - 1
// false   -   2i

bool TwoSat() {
  SCC::SCC(N + N);
  using SCC::code;
  for (int i = 1; i <= N; ++i)
    if (code[i + i - 1] == code[i + i])
      return false;
  for (int i = 1; i <= N; ++i)
    if (code[i + i - 1] > code[i + i])
      // i : selected
    else
      // i : not selected
  return true;
}

\end{lstlisting}

  \subsection{全局最小割 Stoer-Wagner}
    \begin{lstlisting}

const int MAXN = 129;

int minCut(int N, int G[MAXN][MAXN]) { // 0-based
  static int weight[MAXN];
  static bool used[MAXN];

  int ans = INT_MAX;
  while (N > 1) {
    for (int i = 0; i < N; ++i) used[i] = false;
    for (int i = 0; i < N; ++i) weight[i] = G[i][0];
    int S = -1, T = 0;
    for (int _r = 2; _r <= N; ++_r) { // N - 1 selections
      int x = -1;
      for (int i = 0; i < N; ++i) if (!used[i])
        if (x == -1 || weight[i] > weight[x])
          x = i;
      for (int i = 0; i < N; ++i)
        weight[i] += G[x][i];
      S = T; T = x;
      used[x] = true;
    }
    ans = min(ans, weight[T]);
    for (int i = 0; i < N; ++i) {
      G[i][S] += G[i][T];
      G[S][i] += G[i][T];
    }
    G[S][S] = 0;
    for (int i = 0; i < N; ++i) swap(G[i][T], G[i][N]);
    --N;
    for (int i = 0; i < N; ++i) swap(G[T][i], G[N][i]);
  }
  return ans;
}

\end{lstlisting}

  \subsection{Hopcroft-Karp}
    \begin{lstlisting}

namespace HopcroftKarp { // 0-based
  int N, M;
  bool used[MAXN];
  int level[MAXN];
  int matchX[MAXN], matchY[MAXN];

  bool DFS(int x) {
    used[x] = true;
    for (edge e(fir[x]); e; e = e->next) {
      int y = e->to;
      int z = matchY[y];
      if (z == -1 || (!used[z] && level[x] < level[z] && DFS(z))) {
        matchX[x] = y;
        matchY[y] = x;
        return true;
      }
    }
    return false;
  }

  int maxMatch() {
    for (int i = 0; i < N; ++i) used[i] = false;
    for (int i = 0; i < N; ++i) matchX[i] = -1;
    for (int i = 0; i < M; ++i) matchY[i] = -1;
    for (int i = 0; i < N; ++i) level[i] = -1;

    int match = 0;
    for ( ; ; ) {
      static int Q[MAXN * 2 + 1];
      int head = 0, tail = 0;
      for (int x = 0; x < N; ++x) level[x] = -1;
      for (int x = 0; x < N; ++x) if (matchX[x] == -1) {
        level[x] = 0;
        Q[++tail] = x;
      }
      while (head < tail) {
        int x = Q[++head];
        for (edge e(fir[x]); e; e = e->next) {
          int y = e->to;
          int z = matchY[y];
          if (z != -1 && level[z] < 0) {
            level[z] = level[x] + 1;
            Q[++tail] = z;
          }
        }
      }
      for (int x = 0; x < N; ++x) used[x] = false;
      int d = 0;
      for (int x = 0; x < N; ++x) if (matchX[x] == -1) {
        if (DFS(x))
          ++d;
      }
      if (d == 0) break;
      match += d;
    }
    return match;
  }
}



\end{lstlisting}

  \subsection{欧拉路}
    \begin{lstlisting}

vector<int> eulerianWalk(int N, int S) {
  static int res[MAXM];
  static int stack[MAXN];
  static edge cur[MAXN];

  int rcnt = 0, top = 0;
  for (int i = 1; i <= N; ++i) cur[i] = fir[i];
  for (stack[top++] = S; top; ) {
    int x = stack[--top];
    for ( ; ; ) {
      edge &e = cur[x];
      if (e == NULL) break;
      stack[top++] = x;
      x = e->to;
      e = e->next; // the opposite edge must be banned if this is a bidirectional graph
    }
    res[rcnt++] = x;
  }
  reverse(res, res + rcnt);
  return vector<int>(res, res + rcnt);
}

\end{lstlisting}

  \subsection{稳定婚姻}
    \begin{lstlisting}
namespace StableMatching {
  int pairM[MAXN], pairW[MAXN], p[MAXN];
  void stableMatching(int n, int orderM[MAXN][MAXN], int preferW[MAXN][MAXN]) {
    for (int i = 0; i < n; ++i) pairM[i] = -1;
    for (int i = 0; i < n; ++i) pairW[i] = -1;
    for (int i = 0; i < n; i++) {
      while (pairM[i] < 0) {
        int w = orderM[i][p[i]++], m = pairW[w];
        if (m == -1) {
          pairM[i] = w;
          pairW[w] = i;
        } 
        else if (preferW[w][i] < preferW[w][m]) {
          pairM[m] = -1;
          pairM[i] = w;
          pairW[w] = i;
          i = m;
        }
      }
    }
  }
}

\end{lstlisting}

  \subsection{最大团搜索}
    \begin{lstlisting}
namespace MaxClique {
  // 1-based
  // mc[i]: max clique in [i ... n]
  // mc[i] = mc[i + 1] or mc[i + 1] + 1

  const int MAXN = 133;

  int g[MAXN][MAXN];
  int len[MAXN];
  int list[MAXN][MAXN];
  int mc[MAXN];
  int ans;
  bool found;

  void DFS(int size) {
    int i, j, k;
    if (len[size] == 0) {
      if (size > ans) {
        ans = size;
        found = true;
      }
      return;
    }
    for (k = 0; k < len[size] && !found; ++k) {
      if (size + len[size] - k <= ans)
        break;
      i = list[size][k];
      if (size + mc[i] <= ans)
        break;
      for (j = k + 1, len[size + 1] = 0; j < len[size]; ++j)
        if (g[i][list[size][j]])
          list[size + 1][len[size + 1]++] = list[size][j];
      DFS(size + 1);
    }
  }

  int work(int n) {
    int i, j;
    mc[n] = ans = 1;
    for (i = n - 1; i; --i) {
      found = false;
      len[1] = 0;
      for (j = i + 1; j <= n; ++j)
        if (g[i][j])
          list[1][len[1]++] = j;
      DFS(1);
      mc[i] = ans;
    }
    return ans;
  }

}

\end{lstlisting}
  
  \subsection{极大团计数}
    \begin{lstlisting}
namespace MaxCliqueCounting {

  int n, ans;
  int ne[MAXN], ce[MAXN];
  int g[MAXN][MAXN], list[MAXN][MAXN];

  void dfs(int size) {
    int i, j, k, t, cnt, best = 0;
    bool bb;
    if (ne[size] == ce[size]) {
      if (ce[size] == 0)
        ++ans;
      return;
    }
    for (t = 0, i = 1; i <= ne[size]; ++i) {
      for (cnt = 0, j = ne[size] + 1; j <= ce[size]; ++j)
        if (!g[list[size][i]][list[size][j]])
          ++cnt;
      if (t == 0 || cnt < best)
        t = i, best = cnt;
    }
    if (t && best <= 0)
      return;
    for (k = ne[size] + 1; k <= ce[size]; ++k) {
      if (t > 0) {
        for (i = k; i <= ce[size]; ++i)
          if (!g[list[size][t]][list[size][i]])
            break;
        swap(list[size][k], list[size][i]);
      }
      i = list[size][k];
      ne[size + 1] = ce[size + 1] = 0;
      for (j = 1; j < k; ++j)
        if (g[i][list[size][j]])
          list[size + 1][++ne[size + 1]] = list[size][j];
      for (ce[size + 1] = ne[size + 1], j = k + 1; j <= ce[size]; ++j)
        if (g[i][list[size][j]])
          list[size + 1][++ce[size + 1]] = list[size][j];
      dfs(size + 1);
      ++ne[size];
      --best;
      for (j = k + 1, cnt = 0; j <= ce[size]; ++j)
        if (!g[i][list[size][j]])
          ++cnt;
      if (t == 0 || cnt < best)
        t = k, best = cnt;
      if (t && best <= 0)
        break;
    }
  }

  void work() {
    int i;
    ne[0] = 0;
    ce[0] = 0;
    for (i = 1; i <= n; ++i)
      list[0][++ce[0]] = i;
    ans = 0;
    dfs(0);
  }

}

\end{lstlisting}

  \subsection{最小树形图}
    \begin{lstlisting}
#include <iostream>
#include <cstring>
#include <cstdio>
#include <cmath>
#include <algorithm>
using namespace std;

const int INF = 99999999;

struct point {
  double x, y;
} p[200];

int pre[200]; //记录该节点的前驱
double graph[200][200], ans; //图数组和结果
bool visit[110], circle[110]; //visit记录该点有没有被访问过,circle记录改点是不是在一个圈里
int n, m, root; //顶点数+边数+根节点标号

void dfs(int t) { //一个深度优先搜索,搜索出一个最大的联通空间
  int i;
  visit[t] = true;
  for (i = 1; i <= n; ++i) {
    if (!visit[i] && graph[t][i] != INF)
      dfs(i);
  }
}

bool check() { //这个函数用来检查最小树形图是否存在,即如果存在,那么一遍dfs后,应该可以遍历到所有的节点
  memset(visit, false, sizeof(visit));
  dfs(root);

  for (int i = 1; i <= n; ++i)
    if (!visit[i])
      return false;
  return true;
}

double dist(int i, int j) {
  return sqrt(
    (p[i].x - p[j].x) * (p[i].x - p[j].x) + 
    (p[i].y - p[j].y) * (p[i].y - p[j].y));
}

int exist_circle() { //判断图中是不是存在有向圈
  int i;
  int j;
  root = 1;
  pre[root] = root;
  for (i = 1; i <= n; ++i) {
    if (!circle[i] && i != root) {
      pre[i] = i;
      graph[i][i] = INF;

      for (j = 1; j <= n; ++j) {
        if (!circle[j] && graph[j][i] < graph[pre[i]][i])
          pre[i] = j;
      }
    }
  }  //这个for循环负责找出所有非根节点的前驱节点
  for (i = 1; i <= n; ++i) {
    if (circle[i]) continue;
    memset(visit, false, sizeof(visit));
    int j = i;
    while (!visit[j]) {
      visit[j] = true;
      j = pre[j];
    }
    if (j == root)
      continue;
    return j;
  } //找圈过程,最后返回值是圈中的一个点

  return -1; //如果没有圈,返回-1
}

void update(int t) { //缩圈之后更新数据
  int i, j;
  ans += graph[pre[t]][t];
  for (i = pre[t]; i != t; i = pre[i]) {
    ans += graph[pre[i]][i];
    circle[i] = true;
  } //首先把圈里的边权全部加起来,并且留出t节点,作为外部接口

  for (i = 1; i <= n; ++i)
    if (!circle[i] && graph[i][t] != INF)
      graph[i][t] -= graph[pre[t]][t];
  //上面这个for循环的作用是对t节点做更新操作,为什么要单独做?你可以看看线面这个循环的跳出条件。

  for (j = pre[t]; j != t; j = pre[j])
    for (int i = 1; i <= n; ++i) {
      if (circle[i])
        continue;
      if (graph[i][j] != INF)
        graph[i][t] = min(graph[i][t], graph[i][j] - graph[pre[j]][j]);
      /**/ //////////////////////////////////////////////////////////////////////////
      graph[t][i] = min(graph[j][i], graph[t][i]);
    }
  //这个循环对圈中的其他顶点进行更新
}

void solve() {
  int j;
  memset(circle, false, sizeof(circle));
  while ((j = exist_circle()) != -1)
    update(j);

  for (j = 1; j <= n; ++j)
    if (j != root && !circle[j])
      ans += graph[pre[j]][j];

  printf("%.2f\n", ans);
}

int main() {
  int i;
  while (scanf("%d%d", &n, &m) != EOF) {
    for (i = 0; i <= n; ++i)
      for (int j = 0; j <= n; ++j)
        graph[i][j] = INF;

    for (i = 1; i <= n; ++i)
      scanf("%lf%lf", &p[i].x, &p[i].y);

    for (i = 0; i < m; ++i) {
      int a, b;
      scanf("%d%d", &a, &b);
      graph[a][b] = dist(a, b);
    }

    root = 1;
    ans = 0;
    if (!check())
      printf("poor snoopy\n");
    else
      solve();
  }

  return 0;
}

\end{lstlisting}

  \subsection{动态最小生成树}
    \begin{lstlisting}
/*
 * 动态最小生成树
 * Q(logQ)^2
 * (qx[i], qy[i]) 表示将编号为 qx[i] 的边的权值改为 qy[i]
 * 删除一条边相当于将其权值改为 \infinity
 * 加入一条边相当于将其权值从 \infinity 变成某个值
 */

#include <cstdio>
#include <algorithm>
using namespace std;

const int maxn = 100000 + 5;
const int maxm = 1000000 + 5;
const int maxq = 1000000 + 5;
const int qsize = maxm + 3 * maxq;

int x[qsize], y[qsize], z[qsize];
int qx[maxq], qy[maxq];
int n, m, Q;

void init() {
  scanf("%d%d", &n, &m);
  for (int i = 0; i < m; i++)
    scanf("%d%d%d", x + i, y + i, z + i);
  scanf("%d", &Q);
  for (int i = 0; i < Q; i++) {
    scanf("%d%d", qx + i, qy + i);
    qx[i]--;
  }
}

int a[maxn];
int *tz;
int find(int x) {
  int root = x;
  while (a[root]) root = a[root];
  int next;
  while ((next = a[x]) != 0) {
    a[x] = root;
    x = next;
  }
  return root;
}
inline bool cmp(const int &a, const int &b) {
  return tz[a] < tz[b];
}
int kx[maxn], ky[maxn], kt;
int vd[maxn], id[maxm];
int app[maxm];
bool extra[maxm];
long long printState(int *qx, int *qy, int Q, int n, int *x, int *y, int *z,
    int m, long long ans) {
  printf("%d %d\n", n, m);
  for (int i = 0; i < m; i++)
    printf("%d %d %d\n", x[i], y[i], z[i]);
  printf("Q = %d\n", Q);
  for (int i = 0; i < Q; i++)
    printf("%d %d\n", qx[i], qy[i]);
  return ans;
}
void solve(int *qx, int *qy, int Q, int n, int *x, int *y, int *z, int m, long long ans) {
  if (Q == 1) {
    for (int i = 1; i <= n; i++)
      a[i] = 0;
    z[qx[0]] = qy[0];
    for (int i = 0; i < m; i++)
      id[i] = i;
    tz = z;
    sort(id, id + m, cmp);
    int ri, rj;
    for (int i = 0; i < m; i++) {
      ri = find(x[id[i]]);
      rj = find(y[id[i]]);
      if (ri != rj) {
        ans += z[id[i]];
        a[ri] = rj;
      }
    }
    printf("%I64d\n", ans);
    return;
  }
  int ri, rj;
  //contract
  kt = 0;
  for (int i = 1; i <= n; i++)
    a[i] = 0;
  for (int i = 0; i < Q; i++) {
    ri = find(x[qx[i]]);
    rj = find(y[qx[i]]);
    if (ri != rj)
      a[ri] = rj;
  }
  int tm = 0;
  for (int i = 0; i < m; i++)
    extra[i] = true;
  for (int i = 0; i < Q; i++)
    extra[qx[i]] = false;
  for (int i = 0; i < m; i++)
    if (extra[i])
      id[tm++] = i;
  tz = z;
  sort(id, id + tm, cmp);
  for (int i = 0; i < tm; i++) {
    ri = find(x[id[i]]);
    rj = find(y[id[i]]);
    if (ri != rj) {
      a[ri] = rj;
      ans += z[id[i]];
      kx[kt] = x[id[i]];
      ky[kt] = y[id[i]];
      kt++;
    }
  }
  for (int i = 1; i <= n; i++)
    a[i] = 0;
  for (int i = 0; i < kt; i++)
    a[find(kx[i])] = find(ky[i]);
  int n2 = 0;
  for (int i = 1; i <= n; i++)
    if (a[i] == 0)
      vd[i] = ++n2;
  for (int i = 1; i <= n; i++)
    if (a[i])
      vd[i] = vd[find(i)];
  int *Nx = x + m;
  int *Ny = y + m;
  int *Nz = z + m;
  int m2 = 0;
  for (int i = 0; i < m; i++)
    app[i] = -1;
  for (int i = 0; i < Q; i++)
    if (app[qx[i]] == -1) {
      Nx[m2] = vd[x[qx[i]]];
      Ny[m2] = vd[y[qx[i]]];
      Nz[m2] = z[qx[i]];
      app[qx[i]] = m2;
      m2++;
    }
  for (int i = 0; i < Q; i++) {
    z[qx[i]] = qy[i];
    qx[i] = app[qx[i]];
  }
  for (int i = 1; i <= n2; i++)
    a[i] = 0;
  for (int i = 0; i < tm; i++) {
    ri = find(vd[x[id[i]]]);
    rj = find(vd[y[id[i]]]);
    if (ri != rj) {
      a[ri] = rj;
      Nx[m2] = vd[x[id[i]]];
      Ny[m2] = vd[y[id[i]]];
      Nz[m2] = z[id[i]];
      m2++;
    }
  }
  int mid = Q / 2;
  solve(qx, qy, mid, n2, Nx, Ny, Nz, m2, ans);
  solve(qx + mid, qy + mid, Q - mid, n2, Nx, Ny, Nz, m2, ans);
}

void work() {
  if (Q) solve(qx, qy, Q, n, x, y, z, m, 0);
}

int main() {
  init();
  work();
  return 0;
}

\end{lstlisting}

  \subsection{K短路(允许重复)}
    \begin{lstlisting}
//  Author: Amber

#include <cstdio>
#include <cstdlib>
#include <cstring>
#include <vector>
#include <algorithm>
#include <queue>

using namespace std;

#define for_each(it, v) for (vector<Edge*>::iterator it = (v).begin(); it != (v).end(); ++it)

const int MAX_N = 10000;
const int MAX_M = 50000;
const int MAX_K = 10000;
const int INF = 1000000000;

struct Edge {
  int from, to, weight;
};

struct HeapNode {
  Edge* edge;
  int depth;
  HeapNode* child[4];
  //child[0..1] for heap G
  //child[2..3] for heap out edge
};

int n, m, k, s, t;
Edge* edge[MAX_M];
int dist[MAX_N];
Edge* prev[MAX_N];
vector<Edge*> graph[MAX_N];
vector<Edge*> graphR[MAX_N];
HeapNode* nullNode;
HeapNode* heapTop[MAX_N];

HeapNode* createHeap(HeapNode* curNode, HeapNode* newNode) {
  if (curNode == nullNode) return newNode;
  HeapNode* rootNode = new HeapNode;
  memcpy(rootNode, curNode, sizeof(HeapNode));
  if (newNode->edge->weight < curNode->edge->weight) {
    rootNode->edge = newNode->edge;
    rootNode->child[2] = newNode->child[2];
    rootNode->child[3] = newNode->child[3];
    newNode->edge = curNode->edge;
    newNode->child[2] = curNode->child[2];
    newNode->child[3] = curNode->child[3];
  }
  if (rootNode->child[0]->depth < rootNode->child[1]->depth)
    rootNode->child[0] = createHeap(rootNode->child[0], newNode);
  else
    rootNode->child[1] = createHeap(rootNode->child[1], newNode);
  rootNode->depth = max(rootNode->child[0]->depth, rootNode->child[1]->depth) + 1;
  return rootNode;
}

bool heapNodeMoreThan(HeapNode* node1, HeapNode* node2) {
  return node1->edge->weight > node2->edge->weight;
}

int main() {
  scanf("%d%d%d", &n, &m, &k);
  scanf("%d%d", &s, &t);
  s--, t--;
  while (m--) {
    Edge* newEdge = new Edge;
    int i, j, w;
    scanf("%d%d%d", &i, &j, &w);
    i--, j--;
    newEdge->from = i;
    newEdge->to = j;
    newEdge->weight = w;
    graph[i].push_back(newEdge);
    graphR[j].push_back(newEdge);
  }

  //Dijkstra
  queue<int> dfsOrder;

  memset(dist, -1, sizeof(dist));
  typedef pair<int, pair<int, Edge*> > DijkstraQueueItem;
  priority_queue<DijkstraQueueItem, vector<DijkstraQueueItem>, greater<DijkstraQueueItem> > dq;
  dq.push(make_pair(0, make_pair(t, (Edge*) NULL)));
  while (!dq.empty()) {
    int d = dq.top().first;
    int i = dq.top().second.first;
    Edge* edge = dq.top().second.second;
    dq.pop();
    if (dist[i] != -1) continue;
    dist[i] = d;
    prev[i] = edge;
    dfsOrder.push(i);
    for_each(it, graphR[i]) dq.push(make_pair(d + (*it)->weight, make_pair((*it)->from, *it)));
  }

  //Create edge heap
  nullNode = new HeapNode;
  nullNode->depth = 0;
  nullNode->edge = new Edge;
  nullNode->edge->weight = INF;
  fill(nullNode->child, nullNode->child + 4, nullNode);

  while (!dfsOrder.empty()) {
    int i = dfsOrder.front();
    dfsOrder.pop();

    if (prev[i] == NULL)
      heapTop[i] = nullNode;
    else
      heapTop[i] = heapTop[prev[i]->to];

    vector<HeapNode*> heapNodeList;
    for_each(it, graph[i]) {
      int j = (*it)->to;
      if (dist[j] == -1)
        continue;
      (*it)->weight += dist[j] - dist[i];
      if (prev[i] != *it) {
        HeapNode* curNode = new HeapNode;
        fill(curNode->child, curNode->child + 4, nullNode);
        curNode->depth = 1;
        curNode->edge = *it;
        heapNodeList.push_back(curNode);
      }
    }

    if (!heapNodeList.empty()) { //Create heap out
      make_heap(heapNodeList.begin(), heapNodeList.end(), heapNodeMoreThan);
      int size = heapNodeList.size();
      for (int p = 0; p < size; p++) {
        heapNodeList[p]->child[2] = 2 * p + 1 < size ? heapNodeList[2 * p + 1] : nullNode;
        heapNodeList[p]->child[3] = 2 * p + 2 < size ? heapNodeList[2 * p + 2] : nullNode;
      }
      heapTop[i] = createHeap(heapTop[i], heapNodeList.front());
    }
  }

  //Walk on DAG
  typedef pair<long long, HeapNode*> DAGQueueItem;
  priority_queue<DAGQueueItem, vector<DAGQueueItem>, greater<DAGQueueItem> > aq;
  if (dist[s] == -1)
    printf("NO\n");
  else {
    printf("%d\n", dist[s]);
    if (heapTop[s] != nullNode)
      aq.push(make_pair(dist[s] + heapTop[s]->edge->weight, heapTop[s]));
  }
  k--;
  while (k--) {
    if (aq.empty()) {
      printf("NO\n");
      continue;
    }
    long long d = aq.top().first;
    HeapNode* curNode = aq.top().second;
    aq.pop();
    printf("%I64d\n", d);
    if (heapTop[curNode->edge->to] != nullNode)
      aq.push(make_pair(d + heapTop[curNode->edge->to]->edge->weight, heapTop[curNode->edge->to]));
    for (int i = 0; i < 4; i++)
      if (curNode->child[i] != nullNode)
        aq.push(make_pair(d - curNode->edge->weight + curNode->child[i]->edge->weight, curNode->child[i]));
  }

  return 0;
}

\end{lstlisting}

  \subsection{K短路(不允许重复)}
    \begin{lstlisting}

#include <cstdio>
#include <cstring>
#include <vector>
#include <queue>

using namespace std;

int Num[10005][205], Path[10005][205], dev[10005];
int from[10005], value[10005], dist[205], Next[205], Graph[205][205];
bool forbid[205];
bool hasNext[10005][205];
int N, M, K, s, t;
int tot, cnt;

struct cmp {
  bool operator()(const int &a, const int &b) {
    int *i, *j;
    if (value[a] != value[b])
      return value[a] > value[b];
    for (i = Path[a], j = Path[b]; (*i) == (*j); i++, j++);
    return (*i) > (*j);
  }
};

void Check(int idx, int st, int *path, int &res) {
  int i, j;
  for (i = 0; i < N; i++) {
    dist[i] = 1000000000;
    Next[i] = t;
  }
  dist[t] = 0;
  forbid[t] = true;
  j = t;
  while (1) {
    for (i = 0; i < N; i++)
      if (!forbid[i] && (i != st || !hasNext[idx][j]) && (dist[j] + Graph[i][j] < dist[i] || (dist[j] + Graph[i][j] == dist[i] && j < Next[i]))) {
        Next[i] = j;
        dist[i] = dist[j] + Graph[i][j];
      }
    j = -1;
    for (i = 0; i < N; i++)
      if (!forbid[i] && (j == -1 || dist[i] < dist[j])) j = i;
    if (j == -1) break;
    forbid[j] = 1;
    if (j == st) break;
  }
  res += dist[st];
  for (i = st; i != t; i = Next[i], path++) (*path) = i;
  (*path) = i;
}

int main() {
  int i, j, k, l;
  while (scanf("%d%d%d%d%d", &N, &M, &K, &s, &t) && N) {
    priority_queue<int, vector<int>, cmp> Q;
    for (i = 0; i < N; i++)
      for (j = 0; j < N; j++)
        Graph[i][j] = 1000000000;
    for (i = 0; i < M; i++) {
      scanf("%d%d%d", &j, &k, &l);
      Graph[j - 1][k - 1] = l;
    }
    s--;
    t--;
    memset(forbid, false, sizeof(forbid));
    memset(hasNext[0], false, sizeof(hasNext[0]));
    Check(0, s, Path[0], value[0]);
    dev[0] = 0;
    from[0] = 0;
    Num[0][0] = 0;
    Q.push(0);
    cnt = 1;
    tot = 1;
    for (i = 0; i < K; i++) {
      if (Q.empty()) break;
      l = Q.top(); Q.pop();
      for (j = 0; j <= dev[l]; j++)
        Num[l][j] = Num[from[l]][j];
      for (; Path[l][j] != t; j++) {
        memset(hasNext[tot], false, sizeof(hasNext[tot]));
        Num[l][j] = tot++;
      }
      for (j = 0; Path[l][j] != t; j++)
        hasNext[Num[l][j]][Path[l][j + 1]] = true;
      for (j = dev[l]; Path[l][j] != t; j++) {
        memset(forbid, false, sizeof(forbid));
        value[cnt] = 0;
        for (k = 0; k < j; k++) {
          forbid[Path[l][k]] = true;
          Path[cnt][k] = Path[l][k];
          value[cnt] += Graph[Path[l][k]][Path[l][k + 1]];
        }
        Check(Num[l][j], Path[l][j], &Path[cnt][j], value[cnt]);

        if (value[cnt] > 2000000)
          continue;
        dev[cnt] = j;
        from[cnt] = l;
        Q.push(cnt);
        cnt++;
      }
    }

    if (i < K || value[l] > 2000000)
      printf("None\n");
    else {
      for (i = 0; Path[l][i] != t; i++)
        printf("%d-", Path[l][i] + 1);
      printf("%d\n", t + 1);
    }
  }
  return 0;
}

\end{lstlisting}
  
  \subsection{小知识}
  \begin{itemize}
  \item 平面图:
    \begin{enumerate}
    \item 平面图一定存在一个度小于等于$5$的点
    \item $E \le 3V - 6$
    \item 欧拉公式:$V + F - E = 1 + \mbox{连通块数}$
    \end{enumerate}
  \item 图连通度:
    \begin{enumerate}
    \item $k-$连通(\emph{k-connected}):对于任意一对结点都至少存在结点各不相同的$k$条路
    \item 点连通度(\emph{vertex connectivity}):把图变成非连通图所需删除的最少点数
    \item Whitney定理:一个图是$k-$连通的当且仅当它的点连通度至少为$k$
    \end{enumerate}
  \end{itemize}

\section{数学}
  \subsection{单纯形}
    \begin{lstlisting}

// max { cx | Ax ≤ b, x ≥ 0}

double[] simplex(double[][] A, double[] b, double[] c) {
  int n = A.length, m = A[0].length + 1, r = n, s = m - 1;
  double[][] D = new double[n + 2][m + 1];
  int[] ix = new int[n + m];
  for (int i = 0; i < n + m; i++) ix[i] = i;
  for (int i = 0; i < n; i++) {
    for (int j = 0; j < m - 1; j++) D[i][j] = -A[i][j];
    D[i][m - 1] = 1;
    D[i][m] = b[i];
    if (D[r][m] > D[i][m]) r = i;
  }
  for (int j = 0; j < m - 1; j++) D[n][j] = c[j];
  D[n + 1][m - 1] = -1;
  for (double d; ; ) {
    if (r < n) {
      int t = ix[s]; ix[s] = ix[r + m]; ix[r + m] = t;
      D[r][s] = 1.0 / D[r][s];
      for (int j = 0; j <= m; j++) if (j != s) D[r][j] *= -D[r][s];
      for (int i = 0; i <= n + 1; i++) if (i != r) {
        for (int j = 0; j <= m; j++) if (j != s) D[i][j] += D[r][j] * D[i][s];
        D[i][s] *= D[r][s];
      }
    }
    r = -1; s = -1;
    for (int j = 0; j < m; j++) if (s < 0 || ix[s] > ix[j]) {
      if (D[n + 1][j] > EPS || D[n + 1][j] > -EPS && D[n][j] > EPS) s = j;
    }
    if (s < 0) break;
    for (int i = 0; i < n; i++) if (D[i][s] < -EPS) {
      if (r < 0 || (d = D[r][m] / D[r][s] - D[i][m] / D[i][s]) < -EPS
            || d < EPS && ix[r + m] > ix[i + m])
        r = i;
    }
    if (r < 0) return null; // 非有界
  }
  if (D[n + 1][m] < -EPS) return null; // 无法执行
  double[] x = new double[m - 1];
  for (int i = m; i < n + m; i++) if (ix[i] < m - 1) x[ix[i]] = D[i - m][m];
  return x; // 值为 D[n][m]
}


\end{lstlisting}

  \subsection{FFT}
    \begin{lstlisting}
namespace FFT {

  struct Complex {
    double a, b;
    Complex(): a(0.0), b(0.0) {
    }
    Complex(double a, double b): a(a), b(b) {
    }
  };

  int n, id, *A, *B, *s, len;
  Complex tmp[MAXLEN], pa[MAXLEN], pb[MAXLEN], *p;

  void Fill(int m, int d) {
    if (m == n) {
      if (id < len)
        p[d] = Complex(s[id++], 0.0);
      else
        p[d] = Complex();
    }
    else {
      Fill(m << 1, d);
      Fill(m << 1, d + m);
    }
  }

  void Fill2(int m, int d) {
    if (m == n) {
      p[d] = tmp[id++];
    }
    else {
      Fill2(m << 1, d);
      Fill2(m << 1, d + m);
    }
  }

  void FFT(int oper) {
    for (int d = 0; (1 << d) < n; ++d) {
      int m = 1 << d;
      double p0 = PI / m * oper;
      double sinp0 = sin(p0);
      double cosp0 = cos(p0);
      for (int i = 0; i < n; i += m << 1) {
        double sinp = 0, cosp = 1;
        for (int j = 0; j < m; ++j) {
          double ta = cosp * p[i + j + m].a - sinp * p[i + j + m].b;
          double tb = cosp * p[i + j + m].b + sinp * p[i + j + m].a;
          p[i + j + m].a = p[i + j].a - ta;
          p[i + j + m].b = p[i + j].b - tb;
          p[i + j].a += ta;
          p[i + j].b += tb;
          double tsinp = sinp;
          sinp =  tsinp * cosp0 + cosp * sinp0;
          cosp = -tsinp * sinp0 + cosp * cosp0;
        }
      }
    }
  }

  void doFFT(int a[], int la, int b[], int lb, int C[]) {
    A = a; B = b;
    for (n = la + lb; n != Lowbit(n); n += Lowbit(n));

    id = 0; s = A; p = pa; len = la; Fill(1, 0); FFT(1);
    id = 0; s = B; p = pb; len = lb; Fill(1, 0); FFT(1); 
    for (int i = 0; i < n; ++i) {
      tmp[i].a = pa[i].a * pb[i].a - pa[i].b * pb[i].b;
      tmp[i].b = pa[i].b * pb[i].a + pa[i].a * pb[i].b;
    }

    id = 0; p = pa; Fill2(1, 0); FFT(-1);
    for (int i = 0; i < n; ++i) {
      double t = p[i].a / (double)(n);
      C[i] = int(t + 0.5);
    }
  }
}


\end{lstlisting}

  \subsection{整数FFT}
    \begin{lstlisting}

#include <iostream>
#include <cstdio>
#include <vector>

using namespace std;

const int MOD = 3 << 18 | 1;
const int PRIMITIVE_ROOT = 10;
const int MAXB = 1 << 20;

namespace PrimitiveRoot {
//  3 * 2 ^ 21 + 1 is a suitable alternate, PrimitiveRoot = 3
  int powMod(int a, int d, int p) {
    int res = 1;
    a %= p;
    for ( ; d; d >>= 1) {
      if (d & 1)
        res = (long long)res * a % p;
      a = (long long)a * a % p;
    }
    return res;
  }

  bool isPrime(int n) {
    for (int i = 2; i * i <= n; ++i)
      if (n % i == 0)
        return false;
    return true;
  }

  int getMod(int downLimit) {
    for (int c = 3; ; ++c) {
      int t = (c << 21) | 1;
      if (isPrime(t) && t >= downLimit)
        return t;
    }
    return -1;
  }

  bool isPrimitiveRoot(int a, int mod) {
    int phi = mod - 1;
    for (int i = 1; i * i <= phi; ++i) {
      int j = phi / i;
      if (i * j != phi) continue;
      if (i < phi && powMod(a, i, mod) == 1)
        return false;
      if (j < phi && powMod(a, j, mod) == 1)
        return false;
    }
    return true;
  }

  int getPrimitiveRoot(int p) {
    int g = 2;
    while ( !isPrimitiveRoot(g, p) ) ++g;
    return g;
  }
}

namespace FFT {

  int modinv(int a) {
    return a <= 1 ? a : (long long) (MOD - MOD / a) * modinv(MOD % a) % MOD;
  }

  long long powmod(long long a, int b) {
    a %= MOD;
    long long r = 1;
    while (b) {
      if (b & 1) {
        r = r * a % MOD;
      }
      if (b >>= 1) {
        a = a * a % MOD;
      }
    }
    return r;
  }

  void NTT(int P[], int n, int oper) {
    for (int i = 1, j = 0; i < n - 1; i++) {
      for (int s = n; j ^= s >>= 1, ~j & s;);
      if (i < j) {
        swap(P[i], P[j]);
      }
    }
    for (int d = 0; (1 << d) < n; d++) {
      int m = 1 << d, m2 = m * 2;
      long long unit_p0 = powmod(PRIMITIVE_ROOT, (MOD - 1) / m2);
      if (oper < 0) {
        unit_p0 = modinv(unit_p0);
      }
      for (int i = 0; i < n; i += m2) {
        long long unit = 1;
        for (int j = 0; j < m; j++) {
          int &P1 = P[i + j + m], &P2 = P[i + j];
          int t = unit * P1 % MOD;
          P1 = (P2 - t + MOD) % MOD;
          P2 = (P2 + t) % MOD;
          unit = unit * unit_p0 % MOD;
        }
      }
    }
  }

  vector<int> mul(const vector<int> &a, const vector<int> &b) {
    vector<int> ret(max(0, (int) a.size() + (int) b.size() - 1), 0);
    static int A[MAXB], B[MAXB], C[MAXB];
    int len = 1;
    while (len < ret.size()) {
      len *= 2;
    }
    for (int i = 0; i < len; i++) {
      A[i] = i < a.size() ? a[i] : 0;
      B[i] = i < b.size() ? b[i] : 0;
    }
    NTT(A, len, 1);
    NTT(B, len, 1);
    for (int i = 0; i < len; i++) {
      C[i] = (long long) A[i] * B[i] % MOD;
    }
    NTT(C, len, -1);
    int inv = modinv(len);
    for (int i = 0; i < ret.size(); i++) {
      ret[i] = (long long) C[i] * inv % MOD;
    }
    return ret;
  }

}


\end{lstlisting}

  \subsection{线性同余方程}
    \begin{lstlisting}

// Warning: Pay Attention to Integer Overflow !

int gcd(int a, int b) {
  if (a < 0) a = -a;
  if (b < 0) b = -b;
  if (a == 0 || b == 0) return a + b;
  for (int c; (c = a % b) != 0; a = b, b = c);
  return b;
}

int powMod(int a, int d, int MOD) {
  int res = 1;
  for ( ; d; d >>= 1) {
    if (d & 1)
      res = res * a % MOD;
    a = a * a % MOD;
  }
  return res;
}

int modInv(int a, int p) {
  if (a <= 1) return a % p;
  int ret = -(p / a) * modInv(p % a, p) % p;
  return (ret == 0) ? (ret) : (ret + p);
}

bool congruence(int n, int A[], int B[], int M[], int &x, int &m) {
  x = 0;
  m = 1;
  for (int i = 0; i < n; ++i) {
    int a = A[i] * m, b = B[i] - A[i] * x, d = gcd(a, M[i]);
    if (b % d != 0)
      return false;
    x += m * (b / d * modInv(a / d, M[i] / d) % (M[i] / d));
    m *= M[i] / d;
  }
  x %= m;
  return true;
}

\end{lstlisting}

  \subsection{Miller-Rabin 素性测试}
    \begin{lstlisting}

bool test(LL n, int base) {
  LL m = n - 1, ret = 0;
  int s = 0;
  for ( ; m % 2 == 0; ++s)
    m >>= 1;
  ret = pow_mod(base, m, n);
  if (ret == 1 || ret == n - 1)
    return true;
  for (--s; s >= 0; --s) {
    ret = multiply_mod(ret, ret, n);
    if (ret == n - 1)
      return true;
  }
  return false;
}

LL special[7] = {
  1373653LL, 
  25326001LL, 
  3215031751LL, 
  25000000000LL,
  2152302898747LL, 
  3474749660383LL, 
  341550071728321LL
};

/*
 * n < 2047                         test[] = {2}
 * n < 1,373,653                    test[] = {2, 3}
 * n < 9,080,191                    test[] = {31, 73}
 * n < 25,326,001                   test[] = {2, 3, 5}
 * n < 4,759,123,141                test[] = {2, 7, 61}
 * n < 1,122,004,669,633            test[] = {2, 13, 23, 1662803}
 * n < 2,152,302,898,747            test[] = {2, 3, 5, 7, 11}
 * n < 3,474,749,660,383            test[] = {2, 3, 5, 7, 11, 13}
 * n < 341,550,071,728,321          test[] = {2, 3, 5, 7, 11, 13, 17}
 * n < 3,825,123,056,546,413,051    test[] = {2, 3, 5, 7, 11, 13, 17, 19, 23}
 */

bool is_prime(LL n) {
  if (n < 2) return false;
  if (n < 4) return true;

  if (!test(n, 2) || !test(n, 3)) return false;
  if (n < special[0]) return true;

  if (!test(n, 5)) return false;
  if (n < special[1]) return true;

  if (!test(n, 7)) return false;
  if (n == special[2]) return false;
  if (n < special[3]) return true;

  if (!test(n, 11)) return false;
  if (n < special[4]) return true;

  if (!test(n, 13)) return false;
  if (n < special[5]) return true;

  if (!test(n, 17)) return false;
  if (n < special[6]) return true;

  return test(n, 19) && test(n, 23) && test(n, 29) && test(n, 31) && test(n, 37);
}

\end{lstlisting}

  \subsection{多项式求根}
    \input{"Math/Root Of Polynomial.tex"}
  
  \subsection{公式}
    \subsubsection{级数求和}
    \begin{itemize}
    \item
      $\sum_{k=1}^{n}k^3 = (\frac{n(n+1)}{2})^2  $
    \item
      $\sum_{k=1}^{n}k^4 = \frac{n(n+1)(2n+1)(3n^2+3n-1)}{30}  $
    \item
      $\sum_{k=1}^{n}k^5 = \frac{n^2(n+1)^2(2n^2+2n-1)}{12}  $
    \item
      $\sum_{k=1}^{n}k(k+1) = \frac{n(n+1)(n+2)}{3}  $
    \item
      $\sum_{k=1}^{n}k(k+1)(k+2) = \frac{n(n+1)(n+2)(n+3)}{4} $
    \item
      $\sum_{k=1}^{n}k(k+1)(k+2)(k+3) = \frac{n(n+1)(n+2)(n+3)(n+4)}{5} $
    \item
      $\mbox{错排:}D_n = n!(1-\frac{1}{1!}+\frac{1}{2!}-\frac{1}{3!}+\ldots+\frac{(-1)^n}{n!}) = (n-1)(D_{n-2}-D_{n-1})$
    \end{itemize}
    \subsubsection{三角公式}
    \begin{itemize}
    \item
      $\sin(\alpha \pm \beta) = \sin\alpha\cos\beta \pm \cos\alpha\sin\beta $
    \item
      $\cos(\alpha \pm \beta) = \cos\alpha\cos\beta \mp \sin\alpha\sin\beta $
    \item
      $\tan(\alpha \pm \beta) = \frac{\tan\alpha \pm \tan\beta}{1 \mp \tan\alpha\tan\beta} $
    \item
      $\tan\alpha \pm \tan\beta = \frac{\sin(\alpha \pm \beta)}{\cos\alpha\cos\beta} $
    \item
      $\sin\alpha+\sin\beta = 2\sin\frac{\alpha+\beta}{2}\cos\frac{\alpha-\beta}{2} $
    \item
      $\sin\alpha-\sin\beta = 2\cos\frac{\alpha+\beta}{2}\sin\frac{\alpha-\beta}{2} $
    \item
      $\cos\alpha+\cos\alpha = 2\cos\frac{\alpha+\beta}{2}\cos\frac{\alpha-\beta}{2} $
    \item
      $\cos\alpha-\cos\beta = -2\sin\frac{\alpha+\beta}{2}\sin\frac{\alpha-\beta}{2} $
    \item
      $\cos{n\alpha} =
          \binom{n}{0} \cos ^ {n} \alpha
        - \binom{n}{2} \cos ^ {n - 2} \alpha \sin ^ {2} \alpha
        + \binom{n}{4} \cos ^ {n - 4} \alpha \sin ^ {4} \alpha
      \cdots $
    \item
      $\sin{n\alpha} =
          \binom{n}{1} \cos ^ {n - 1} \alpha \sin \alpha
        - \binom{n}{3} \cos ^ {n - 3} \alpha \sin ^ {3} \alpha
        + \binom{n}{5} \cos ^ {n - 5} \alpha \sin ^ {5} \alpha
        \cdots $
    \end{itemize}
    
    \subsubsection{三次方程求根公式}
      对一元三次方程
      $x ^ 3 + px + q = 0$,
      令
      \begin{align*}
        A &= \sqrt[3]{-\frac{q}{2}+\sqrt{(\frac{q}{2})^2+(\frac{p}{3})^3}} \\
        B &= \sqrt[3]{-\frac{q}{2}-\sqrt{(\frac{q}{2})^2+(\frac{p}{3})^3}} \\ 
        \omega &= \frac{(-1 + \mathrm{i} \sqrt{3})}{2}
      \end{align*}
      
      则 $x_j = A\omega^{j} + B\omega^{2j}$ (j = 0, 1, 2).
      
      当求解 $ax ^ 3 + bx ^ 2 + cx + d = 0$ 时, 令$x = y - \frac{b}{3a}$, 再求解$y$, 即转化为$y^3 + py + q = 0$ 的形式. 
      其中, 
      \begin{align*}
        p &= \frac{b^2 - 3ac}{3a^2} \\
        q &= \frac{2b ^ 3 - 9 abc + 27 a ^ 2 d}{27 a ^ 3}
      \end{align*}
    
    \subsubsection{抛物线}
    \begin{itemize}
      \item 标准方程$y^2 = 2px$, 曲率半径$ R = \dfrac{(p + 2x)^{\frac{3}{2} }}{\sqrt{p}}$
      \item 弧长: 设$M(x, y)$是抛物线上一点, 则$L_{OM} = \frac{p}{2} [ \sqrt{\frac{2x}{p}(1 + \frac{2x}{p})} + \ln(\sqrt{\frac{2x}{p}} + \sqrt{1 + \frac{2x}{p}})]$
      \item 弓形面积: 设$M, D$是抛物线上两点, 且分居一, 四象限. 做一条平行于$MD$且与抛物线相切的直线$L$. 若$M$到$L$的距离为$h$. 则有$S_{MOD} = \frac{2}{3}MD \cdot h$.
    \end{itemize}

    \subsubsection{重心}
    \begin{itemize}
      \item 半径$r$, 圆心角为$\theta$的扇形的重心与圆心的距离为$\dfrac{4r\sin\frac{\theta}{2}}{3\theta}$
      \item 半径$r$, 圆心角为$\theta$的圆弧的重心与圆心的距离为$\dfrac{4r\sin^3\frac{\theta}{2}}{3(\theta - \sin\theta)}$
      \item 椭圆上半部分的重心与圆心的距离为$\dfrac{4b}{3\pi}$
      \item  抛物线中弓形$MOD$的重心满足$CQ = \frac{2}{5} PQ$, $P$是直线$L$与抛物线的切点, $Q$在$MD$上且$PQ$平行$x$轴, $C$是重心
    \end{itemize}

    \subsubsection{向量恒等式}
    \begin{itemize}
      \item $\vec{a} \cdot (\vec{b} \times \vec{c}) = \vec{b} \cdot (\vec{c} \times \vec{a}) = \vec{c} \cdot (\vec{a} \times \vec{b})$
      \item $\vec{a} \times (\vec{b} \times \vec{c}) = (\vec{c} \times \vec{b}) \times \vec{a} = \vec{b}(\vec{a} \cdot \vec{c}) - \vec{c}(\vec{a} \cdot \vec{b})$
    \end{itemize}

    \subsubsection{平面几何常用公式}
    \begin{itemize}
    \item 四边形: 设$D_1, D_2$为对角线, $M$为对角线中点连线, $A$为对角线夹角
      \begin{itemize}
      \item $a^2 + b^2 + c^2 + d^2 = D_1^2 + D_2 ^ 2 + 4 M^2$
      \item $S = \frac{1}{2} D_1  D_2 \sin A$
      \item $ac + bd = D_1 D_2$ (内接四边形适用)
      \item $S = \sqrt{(p - a)(p - b)(p - c)(p - d)}, p$为半周长 (内接四边形适用)
      \end{itemize}
    \item 棱台:
      \begin{itemize}
      \item 体积$V = \dfrac{(A_1 + A_2 + \sqrt{A_1 A_2}) \cdot h}{3}$, $A_1, A_2$分别为上下底面面积, $h$为高
      \item 侧面积$S = \frac{1}{2} (p_1 + p_2) \cdot l$, $p_1, p_2$为上下底面周长, $l$为斜高. (正棱台适用)
      \end{itemize}
    \end{itemize}

    \subsubsection{树的计数}
    \begin{itemize}
      \item 有根数计数: 令$S_{n, j} = \sum\limits_{1 \le i \le n / j} a_{n + 1 - ij} = S_{n - j, j} + a_{n + 1 - j}$\\
        于是, $n + 1$个结点的有根数的总数为$a_{n + 1} = \dfrac{\sum\limits_{1 \le j \le n} j \cdot a_j \cdot S_{n, j} }{n}$\\
        附: $a_1 = 1, a_2 = 1, a_3 = 2, a_4 = 4, a_5 = 9, a_6 = 20, a_9 = 286, a_{11} = 1842$
      \item 无根树计数: 当$n$是奇数时, 则有$a_n - \sum\limits_{1 \le i \le \frac{n}{2}} a_i a_{n - i}$种不同的无根树\\
        当$n$是偶数时, 则有$a_n - \sum\limits_{1 \le i \le \frac{n}{2}} a_i a_{n - i} + \dfrac{1}{2} a_\frac{n}{2} (a_\frac{n}{2} + 1)$种不同的无根树
      \item Matrix-Tree定理: 对任意图$G$, 设mat[$i$][$i$] = $i$的度数, mat[$i$][$j$] = $i$与$j$之间边数的相反数, 则mat[$i$][$j$]的任意余子式的行列式就是该图的生成树个数
    \end{itemize}

  \subsection{小知识}
  \begin{itemize}
  \item 勾股数:设正整数$n$的质因数分解为$n = \prod p_i ^ {a_i}$, 
    则$x^2+y^2=n$有整数解的充要条件是$n$中不存在形如$p_i \equiv 3\pmod{4}$且指数$a_i$为奇数的质因数$p_i$
  \item 勾股数2:
    \begin{verbatim}
      a[0] := 0;
      s := 0;
      for i := 1 to n - 2 do
        begin
          a[i] := a[i - 1] + 1;
          s := s + sqr(a[i]);
        end;
      {======s + sqr(a[n-1]) + sqr(a[n]) = k^2=======}
      a[n - 1] := a[n - 2];
      repeat
        a[n - 1] := a[n - 1] + 1;
      until odd(s + sqr(a[n - 1])) and (a[n - 1] > 2);
      a[n] := (s + sqr(a[n - 1]) - 1) shr 1;
    \end{verbatim}

    知道\texttt{s}和\texttt{a[n-1]}后, 直接求了\texttt{a[n]}. 神奇了点. 

    其实. 有当$n$为奇数:$n^2+{\lfloor\frac{n^2-1}{2}\rfloor}^2={\lfloor\frac{n^2+1}{2}\rfloor}^2$

    若:

    $a=k\cdot(s^2 - t^2)$

    $b=2\cdot k\cdot s\cdot t$

    $c=k\cdot(s^2 + t^2)$

    则$c^2=a^2+b^2$. 

  \item Pick定理:简单多边形, 不自交. 则:$\frac{(\mbox{严格在多边形内部的整点数}\times 2+\mbox{在边上的整点数}-2)}{2}=\mbox{面积}$
  \item Mersenne素数:$p$是素数且$2^p-1$的数是素数. (10000以内的$p$有: 2, 3, 5, 7, 13, 17, 19, 31, 61, 89, 107, 127, 521, 607, 1279, 2203, 2281, 3217, 4253, 4423, 9689, 9941)
  \item 序列差分表: 差分表的第$0$条对角线确定原序列. 
      设原序列为$h_i$, 第$0$条对角线为$c_0,c_1,\ldots,c_p,0,0,\ldots$. 
      有这样两个公式:
      $h_n = \binom{n}{0}c_0 + \binom{n}{1}c_1 + \ldots + \binom{n}{p} c_p$, 
      $\sum_{k = 0}^{n}h_k = \binom{n+1}{1}c_0 + \binom{n+1}{2}c_2 + \ldots + \binom{n+1}{p+1}c_p$
  \item Fibonacci数相关:
    \begin{itemize}
    \item $\gcd(F_n,F_m)=F_{\gcd(n,m)}$
    \item 如果$a$是$b$的倍数, 那么$F_a$是$F_b$的倍数
    \end{itemize}
  \item GCD:
    $\gcd(2^a-1,2^b-1)=2^{\gcd(a,b)}-1$
  \item Fermat分解算法: 
    从$t=\sqrt{n}$开始, 
    依次检查$t^2-n,(t+1)^2-n,(t+2)^2-n,\ldots$, 
    直到出现一个平方数$y$, 
    由于$t ^ 2 - y ^ 2 = n$, 
    因此分解得$n = (t -y)(t + y)$. 
    显然, 当两个因数很接近时这个方法能很快找到结果, 
    但如果遇到一个素数, 则需要检查$\frac{n + 1}{2} - \sqrt{n}$个整数
  \end{itemize}

\section{其他}
  \subsection{Extended LIS}
    \begin{lstlisting}
int G[MAXN][MAXN];

void insertYoung(int v) {
  for (int x = 1, y = INT_MAX; ; ++x) {
    Down(y, *G[x]);
    while (y > 0 && G[x][y] >= v) --y;
    if (++y > *G[x]) {
      ++*G[x];
      G[x][y] = v;
      break;
    }
    else swap(G[x][y], v);
  }
}

int solve(int N, int seq[]) {
  Rep(i, 1, N) *G[i] = 0;
  Rep(i, 1, N) insertYoung(seq[i]);
  printf("%d\n", *G[1] + *G[2]);
  return 0;
}

\end{lstlisting}

  \subsection{生成 nCk}
    \begin{lstlisting}

void nCk(int n, int k) {
  for (int comb = (1 << k) - 1; comb < (1 << n); ) {
    // ...
    {
      int x = comb & -comb, y = comb + x;
      comb = (((comb & ~y) / x) >> 1) | y;
    }
  }
}

\end{lstlisting}

  \subsection{nextPermutation}
    \begin{lstlisting}
boolean nextPermutation(int[] is) {
  int n = is.length;
  for (int i = n - 1; i > 0; i--) {
    if (is[i - 1] < is[i]) {
      int j = n;
      while (is[i - 1] >= is[--j]);
      swap(is, i - 1, j); // swap is[i - 1], is[j]
      rev(is, i, n); // reverse is[i, n)
      return true;
    }
  }
  rev(is, 0, n);
  return false;
}

\end{lstlisting}

  \subsection{Josephus 数与逆 Josephus 数}
    \begin{lstlisting}
int josephus(int n, int m, int k) {
  int x = -1;
  for (int i = n - k + 1; i <= n; i++) {
    x = (x + m) % i;
  }
  return x;
}

int invJosephus(int n, int m, int x) {
  for (int i = n; ; i--) {
    if (x == i) return n - i;
    x = (x - m % i + i) % i;
  }
}

\end{lstlisting}
  
  \subsection{表达式求值}
    \input{"Others/Expression Evaluation.tex"}

  \subsection{曼哈顿最小生成树}
    \input{"Others/Manhattan MST.tex"}

\section{Templates}
  \subsection{vimrc}
    \begin{lstlisting}

syntax on
syntax enable
set nocompatible
set backspace=indent,eol,start
set autoindent
set mouse=a
set tabstop=4
set shiftwidth=4
set ai
set number
set hlsearch incsearch
set guifont=Consolas:h16
set whichwrap=b,s,<,>,[,]
au GUIEnter * simalt ~x

set nu mouse=a nobk hls ai si ts=4 sts=4 sw=4 foldmethod=marker
"set nowrap, et
"set t_Co=256
set foldmethod=marker

nmap <C-A> ggVG
vmap <C-C> "+y

set fileencodings=ucs-bom,utf-8,cp936,gb18030,gb2312,gbk,big5,euc-jp,euc-kr,latin1

autocmd BufRead,BufNewFile *.cpp,*.java,*.tex,*.py,*.cc,*.h map<F4> : !gedit % <CR>
autocmd BufRead,BufNewFile *.cpp,*.java,*.py,*.cc map<F3> : vnew %<.in <CR>

autocmd BufRead,BufNewFile *.h map<F9> : !g++ -Wall -g3 test.cpp -o test <CR>
"autocmd BufNewFile make_data.cpp 0r $HOME/Template/make_data.cpp
autocmd BufNewFile *.cpp 0r $HOME/Template/cpp_template.cpp
"autocmd BufRead,BufNewFile *.cpp,*.h set foldmethod=indent foldlevel=99
autocmd BufRead,BufNewFile *.cpp,*.cc map<F12> : !g++ -Wall -g3 % -o %< -O2<CR>
autocmd BufRead,BufNewFile *.cpp,*.cc map<F9> : !g++ -Wall -g3 % -o %< <CR>
autocmd BufRead,BufNewFile *.cpp,*.cc map<F8> : !time ./%< < %<.in <CR>
autocmd BufRead,BufNewFile *.cpp,*.cc map<F5> : !time ./%< <CR>
autocmd BufRead,BufNewFile *.cpp,*.h set cindent

autocmd BufNewFile *.java 0r $HOME/Template/java_template.java
autocmd BufRead,BufNewFile *.java map<F9> : !javac % <CR>
autocmd BufRead,BufNewFile *.java map<F8> : !java %< < %<.in <CR>
autocmd BufRead,BufNewFile *.java map<F5> : !java %< <CR>

autocmd BufNewFile *.tex 0r $HOME/Template/TeXTemplate.tex
autocmd BufRead,BufNewFile *.tex map<F9>	: !pdflatex % <CR>
autocmd BufRead,BufNewFile *.tex map<F10>	: !latex % <CR>
autocmd BufRead,BufNewFile *.tex map<F11>	: !dvipdfm %<.dvi <CR>
autocmd BufRead,BufNewFile *.tex map<F12>	: !xelatex % <CR>
autocmd BufRead,BufNewFile *.tex map<F8>	: !okular %<.pdf <CR>
autocmd BufRead,BufNewFile *.tex map<F5>	: !google-chrome %<.pdf <CR>

autocmd BufRead,BufNewFile *.pas map<F9> : !fpc -g % <CR>

autocmd BufNewFile *.sh 0r $HOME/Template/shellScript.sh

autocmd BufNewFile *.py 0r $HOME/Template/pythonScript.py
autocmd BufRead,BufNewFile *.py set foldmethod=indent et sta sw=4 sts=4 foldlevel=99
autocmd BufRead,BufNewFile *.py map<F9> : !chmod +x % <CR>
autocmd BufRead,BufNewFile *.py map<F8> : !./% < %<.in <CR>
autocmd BufRead,BufNewFile *.py map<F5> : !./% <CR>

autocmd BufRead,BufNewFile *.markdown set sw=8 sts=8 ts=8 sta
autocmd BufRead,BufNewFile *.markdown map<F9> : !markdown % >%<.html <CR>


\end{lstlisting}

  \subsection{C++}
    \begin{lstlisting}
#pragma comment(linker, "/STACK:10240000")

#define UseBuffer

#include <cassert>
#include <cctype>
#include <climits>
#include <cmath>
#include <cstdio>
#include <cstdlib>
#include <cstring>
#include <ctime>

#include <algorithm>
#include <bitset>
#include <deque>
#include <functional>
#include <iostream>
#include <list>
#include <map>
#include <numeric>
#include <queue>
#include <sstream>
#include <string>
#include <stack>
#include <set>
#include <utility>
#include <vector>

#define Lowbit(x) ((x) & (-(x)))
#define Pow2(x) (1 << (x))
#define Pow2LL(x) (1LL << (x))
#define Contain(a, x) (((a) >> (x)) & 1)

#define Rep(i, a, b) for(int i = (a); i <= (b); ++i)
#define Foru(i, a, b) for(int i = (a); i < (b); ++i)

#define Debug(x) (cerr << #x << " = " << (x) << endl)
#define Debug2(x, y) (cerr << #x << " = " << (x) << ", " << #y << " = " << (y) << endl)

using namespace std;

typedef long long LL;
typedef pair<int, int> pii;

const int INF = 1000000000;

template <class T> inline bool cMin(T &a, const T &b) {
  return a > b ? (a = b, true) : false;
}

template <class T> inline bool cMax(T &a, const T &b) {
  return a < b ? (a = b, true) : false;
}

#ifdef UseBuffer
#define MaxBuffer 5

namespace BufferedReader {

  char buff[MaxBuffer + 5], *buf = buff;
  char c;
  bool flag;

  inline bool nextChar(char &c) {
    if ( (c = *buf++) == 0 ) {
      int tmp = fread(buff, 1, MaxBuffer, stdin);
      buff[tmp] = 0;
      if (tmp == 0)
        return false;
      buf = buff;      
      c = *buf++;
    }
    return true;
  }

  inline bool nextUnsignedInt(unsigned int &x) {
    for ( ; ; ) {
      if ( !nextChar(c) )
        return false;
      if ('0' <= c && c <= '9') break;
    }
    x = c - '0';
    for ( ; ; ) {
      if ( !nextChar(c) )
        break;
      if (c < '0' || c > '9') break;
      x = x * 10 + c - '0';
    }
    return true;
  }

  inline bool nextInt(int &x) {
    for ( ; ; ) {
      if ( !nextChar(c) )
        return false;
      if (c == '-' || ('0' <= c && c <= '9'))
        break;
    }
    if (c == '-') {
      x = 0;
      flag = true;
    }
    else {
      x = c - '0';
      flag = false;
    }
    for ( ; ; ) {
      if ( !nextChar(c) )
        break;
      if (c < '0' || c > '9') break;
      x = x * 10 + c - '0';
    }
    if (flag) x = -x;
    return true;
  }

};
#endif

int main() {
  int x;
  while (BufferedReader::nextInt(x))
    printf("%d\n", x);
  return 0;
}

\end{lstlisting}
  \subsection{Java}
    \begin{lstlisting}

import java.io.InputStreamReader;
import java.io.IOException;
import java.io.BufferedReader;
import java.io.OutputStream;
import java.io.PrintWriter;
import java.io.InputStream;
import java.util.StringTokenizer;

public class Main {

  public void solve() {

  }

  public void run() {
    tokenizer = null;
    reader = new BufferedReader(new InputStreamReader(System.in));
    out = new PrintWriter(System.out);

    solve();

    out.close();
  }

  public static void main(String[] args) {
    new Main().run();
  }

  public StringTokenizer tokenizer;
  public BufferedReader reader;
  public PrintWriter out;

  public String next() {
    while (tokenizer == null || !tokenizer.hasMoreTokens()) {
      try {
        tokenizer = new StringTokenizer(reader.readLine());
      }
      catch (IOException e) {
        throw new RuntimeException(e);
      }
    }
    return tokenizer.nextToken();
  }

  public int nextInt() {
    return Integer.parseInt(next());
  }

}

\end{lstlisting}
  \subsection{对拍}
    \begin{lstlisting}
make gen
make B_pai
make B_re

$i
while [ true ]
do
  echo $((i=i+1))
  ./gen > B.in
  ./B_pai < B.in > B_pai.out
  ./B_re < B.in > B_re.out
  if ! diff B_pai.out B_re.out
  then
    echo Wrong
    break
  fi
done


\end{lstlisting}

\end{document}

