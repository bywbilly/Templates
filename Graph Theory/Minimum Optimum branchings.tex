\begin{lstlisting}
#include <iostream>
#include <cstring>
#include <cstdio>
#include <cmath>
#include <algorithm>
using namespace std;

const int INF = 99999999;

struct point {
  double x, y;
} p[200];

int pre[200]; //记录该节点的前驱
double graph[200][200], ans; //图数组和结果
bool visit[110], circle[110]; //visit记录该点有没有被访问过,circle记录改点是不是在一个圈里
int n, m, root; //顶点数+边数+根节点标号

void dfs(int t) { //一个深度优先搜索,搜索出一个最大的联通空间
  int i;
  visit[t] = true;
  for (i = 1; i <= n; ++i) {
    if (!visit[i] && graph[t][i] != INF)
      dfs(i);
  }
}

bool check() { //这个函数用来检查最小树形图是否存在,即如果存在,那么一遍dfs后,应该可以遍历到所有的节点
  memset(visit, false, sizeof(visit));
  dfs(root);

  for (int i = 1; i <= n; ++i)
    if (!visit[i])
      return false;
  return true;
}

double dist(int i, int j) {
  return sqrt(
    (p[i].x - p[j].x) * (p[i].x - p[j].x) + 
    (p[i].y - p[j].y) * (p[i].y - p[j].y));
}

int exist_circle() { //判断图中是不是存在有向圈
  int i;
  int j;
  root = 1;
  pre[root] = root;
  for (i = 1; i <= n; ++i) {
    if (!circle[i] && i != root) {
      pre[i] = i;
      graph[i][i] = INF;

      for (j = 1; j <= n; ++j) {
        if (!circle[j] && graph[j][i] < graph[pre[i]][i])
          pre[i] = j;
      }
    }
  }  //这个for循环负责找出所有非根节点的前驱节点
  for (i = 1; i <= n; ++i) {
    if (circle[i]) continue;
    memset(visit, false, sizeof(visit));
    int j = i;
    while (!visit[j]) {
      visit[j] = true;
      j = pre[j];
    }
    if (j == root)
      continue;
    return j;
  } //找圈过程,最后返回值是圈中的一个点

  return -1; //如果没有圈,返回-1
}

void update(int t) { //缩圈之后更新数据
  int i, j;
  ans += graph[pre[t]][t];
  for (i = pre[t]; i != t; i = pre[i]) {
    ans += graph[pre[i]][i];
    circle[i] = true;
  } //首先把圈里的边权全部加起来,并且留出t节点,作为外部接口

  for (i = 1; i <= n; ++i)
    if (!circle[i] && graph[i][t] != INF)
      graph[i][t] -= graph[pre[t]][t];
  //上面这个for循环的作用是对t节点做更新操作,为什么要单独做?你可以看看线面这个循环的跳出条件。

  for (j = pre[t]; j != t; j = pre[j])
    for (int i = 1; i <= n; ++i) {
      if (circle[i])
        continue;
      if (graph[i][j] != INF)
        graph[i][t] = min(graph[i][t], graph[i][j] - graph[pre[j]][j]);
      /**/ //////////////////////////////////////////////////////////////////////////
      graph[t][i] = min(graph[j][i], graph[t][i]);
    }
  //这个循环对圈中的其他顶点进行更新
}

void solve() {
  int j;
  memset(circle, false, sizeof(circle));
  while ((j = exist_circle()) != -1)
    update(j);

  for (j = 1; j <= n; ++j)
    if (j != root && !circle[j])
      ans += graph[pre[j]][j];

  printf("%.2f\n", ans);
}

int main() {
  int i;
  while (scanf("%d%d", &n, &m) != EOF) {
    for (i = 0; i <= n; ++i)
      for (int j = 0; j <= n; ++j)
        graph[i][j] = INF;

    for (i = 1; i <= n; ++i)
      scanf("%lf%lf", &p[i].x, &p[i].y);

    for (i = 0; i < m; ++i) {
      int a, b;
      scanf("%d%d", &a, &b);
      graph[a][b] = dist(a, b);
    }

    root = 1;
    ans = 0;
    if (!check())
      printf("poor snoopy\n");
    else
      solve();
  }

  return 0;
}

\end{lstlisting}